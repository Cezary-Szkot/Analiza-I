\documentclass{article}

\usepackage[utf8]{inputenc}
\usepackage{geometry} \geometry{margin=85pt,tmargin=55pt}
\usepackage[polish]{babel}
\usepackage[OT4]{fontenc}
\usepackage[document]{ragged2e}
\usepackage{amsmath}
\usepackage{amsthm}
\usepackage{amsfonts}
\usepackage{mathtools}
\usepackage{fancyhdr}
\usepackage{enumitem}
\usepackage[skins]{tcolorbox}
\usepackage{kpfonts}
\usepackage[T1]{fontenc}
\usepackage{xcolor}
\usepackage{hyperref}
\usepackage{graphicx}
\usepackage{braket}

%%%%%%%%%%%%%%%
% TWIERDZENIA $
%%%%%%%%%%%%%%%

\newtheorem*{dedekind}{{\color{darkred} Aksjomat ciągłości (Dedekinda)}}

%%%%%%%%%%
% KOLORY %
%%%%%%%%%%

\definecolor{col1}{HTML}{ffcccc}
\definecolor{col2}{HTML}{ccccff}
\definecolor{darkred}{HTML}{8B0000}
\definecolor{goldenrod}{HTML}{FFDF42}
\definecolor{lightgoldenrod}{HTML}{fff0a5}

%%%%%%%%%
% RAMKI %
%%%%%%%%%

\newtcolorbox{defr}[2][]{%
  enhanced,colback=white,colframe=col1,coltitle=black,
  sharp corners,boxrule=1.5pt,
  fonttitle=\bfseries,top=13pt,
  attach boxed title to top left={yshift=-\tcboxedtitleheight/2, xshift=10pt},
  boxed title style={tile,size=small,left=5pt,right=5pt, 
  colback=col1,before upper=\strut},
  title=#2,#1}

\newtcolorbox{twier}[2][]{%
  enhanced,colback=white,colframe=col2,coltitle=black,
  sharp corners,boxrule=1.5pt,
  fonttitle=\bfseries,top=13pt,
  attach boxed title to top left={yshift=-\tcboxedtitleheight/2, xshift=10pt},
  boxed title style={tile,size=small,left=5pt,right=5pt, 
  colback=col2,before upper=\strut},
  title=#2,#1}

\newtcolorbox{wn}[2][]{%
  enhanced,colback=white,colframe=white,coltitle=black,
  sharp corners,boxrule=1.5pt,
  fonttitle=\bfseries,top=13pt,
  attach boxed title to top left={yshift=-\tcboxedtitleheight/2, xshift=0pt},
  boxed title style={tile,size=small,left=5pt,right=5pt, 
  colback=lightgoldenrod,before upper=\strut},
  title=#2,#1}

%%%%%%%%%%%
% KOMENDY %
%%%%%%%%%%%

\newcommand{\hquad}{\mkern9mu}
\newcommand{\R}{\mathbb{R}}
\newcommand{\N}{\mathbb{N}}
\newcommand{\Q}{\mathbb{Q}}
\newcommand{\Po}{\mathcal{P}}
\newcommand{\ifff}{\Leftrightarrow}
\newcommand{\imp}{\Rightarrow}
\newcommand{\ilorazowy}[1]{#1/\!{_R}}
\newcommand{\lin}[1]{\lim\limits_{n\to\infty}{#1}}
\newcommand{\arn}{\xrightarrow{n\to\infty}}
\newcommand{\ar}{\rightarrow}

\ExplSyntaxOn
\NewDocumentCommand{\op}{m}
 {
  \langle
  \clist_set:Nn \l_tmpa_clist { #1 }
  \clist_use:Nn \l_tmpa_clist {,\mspace{3mu plus 1mu minus 1mu}\allowbreak}
  \rangle
}
\ExplSyntaxOff
\newcommand{\seq}[1]{\set{#1_n}_{n\in\N}}


                                \begin{document}


%%%%%%%%%%%%%%%%%%%%%%%%%%%%%%%%%%%%%%%%%%%%%%%%%%%%%%%%%%%%%%%%%%%%%%%%%%%%%%%%%%%%%%%%%%%%%%%%%%%%%%%%%%%%%%%%
                                \section*{Sekcja 1} \smallskip
                                {\Huge\bfseries Relacje} \bigskip \medskip
%%%%%%%%%%%%%%%%%%%%%%%%%%%%%%%%%%%%%%%%%%%%%%%%%%%%%%%%%%%%%%%%%%%%%%%%%%%%%%%%%%%%%%%%%%%%%%%%%%%%%%%%%%%%%%%%

\begin{defr}{Definicja 1.1: Relacja}
    Dane są dwa zbiory $A$ i $B$. \textbf{Relacją (dwuargumentową)} $R$ między elementami zbioru $A$ a elementami 
    zbioru $B$ nazywamy dowolny podzbiór iloczynu kartezjańskiego $A\times B$ ($R\subset A\times B$).  
\end{defr}

Mówimy, że elementy $a\in A$ oraz $b\in B$ są ze sobą w relacji $R$ (ozn. $a\sim b$ lub $aRb$), 
jeśli $\op{a,b}\in R$. \medskip

Niech $R$ będzie relacją na niepustym zbiorze $A$. Mówimy, że:
\begin{enumerate}[label=(\arabic*), nosep]
    \item R jest \textbf{zwrotna} $\ifff (\forall a\in A)\ aRa$.
    \item R jest \textbf{przeciwzwrotna} $\ifff(\forall a\in A)\ \lnot aRa$.
    \item R jest \textbf{przechodnia} $\ifff(\forall a,b,c\in A)(aRb \land bRc \imp aRc)$.
    \item R jest \textbf{symetryczna} $\ifff(\forall a,b\in A)(aRb\imp bRa)$.
    \item R jest \textbf{słabo antysymetryczna} $\ifff(\forall a,b\in A)(aRb\land bRa\imp a=b)$.
    \item R jest \textbf{silnie antysymetryczna} $\ifff(\forall a,b\in A)\ \lnot(aRb\land bRa)\ifff(\forall a,b\in A)
    (aRb\imp\lnot bRa)$.
    \item R jest \textbf{spójna} $\ifff(\forall a,b\in A)(aRb\lor bRa)$.
\end{enumerate} \smallskip

Relację silnie antysymetryczną nazywamy również relacją asymetryczną bądź to
przeciwsymetryczną.

\begin{defr}{Definicja 1.2: Relacja równoważności}
    Niech $R\subset A\times A$. Gdy relacja $R$ jest \textbf{zwrotna}, \textbf{symetryczna} i \textbf{przechodnia}, to mówimy, że
    jest \textbf{relacją równoważności}.
\end{defr}

\begin{defr}{Definicja 1.3: Klasa równoważności}
    Niech $R$ będzie relacją równoważności na zbiorze $A$.\textbf{ Klasą równoważności (abstrakcji)} elementu $a\in A$ względem
    relacji $R$ nazywamy zbiór
    \begin{equation*}
        [a]_R=\set{x\in A:xRa}.
    \end{equation*}
\end{defr}

\begin{defr}{Definicja 1.4: Zbiór ilorazowy}
    Zbiór wszystkich klas abstrakcji (względem) relacji równoważności $R$, czyli zbiór
    \begin{equation*}
        \ilorazowy{A}=\set{[a]_R:a\in A},
    \end{equation*}
    nazywamy \textbf{zbiorem ilorazowym} relacji $R$.
\end{defr}

\begin{twier}{Twierdzenia o klasach równoważności 1.1}
    Niech $R$ będzie relacją równoważności na zbiorze $A$. Wówczas mamy:
    \begin{enumerate}[label=(\roman*)]
        \item $(\forall a\in A)\ a\in [a]_R$ $\leftarrow$ ze zwrotności $R$
        \item $(\forall a,b\in A)(a\in [b]_R\ifff b\in [a]_R)$ $\leftarrow$ z symetryczności $R$
        \item $(\forall a,b\in A)(a\in [b]_R\ifff [a]_R=[b]_R)$
    \end{enumerate}
\end{twier}

\begin{defr}{Definicja 1.5: Relacje porządku częściowego}
    Relację $\preceq$ na zbiorze $A$ nazywamy \textbf{porządkiem częściowym 
    \textcolor{darkred}{słabym (nieostrym)}} na zbiorze $A$, jeśli jest \textbf{zwrotna},
    \textbf{przechodnia} i \textbf{słabo antysymetryczna}. \\ \smallskip
    Relację $\prec$ na zbiorze $A$ nazywamy \textbf{porządkiem częściowym \textcolor{darkred}{ostrym}} na
    zbiorze $A$, jeśli jest \textbf{przeciwzwrotna} i \textbf{przechodnia}.
\end{defr}

Na wykładzie stwierdzono, iż ostry porządek częściowy jest również asymetryczny. Jednak fakt ten wynika
już z przeciwzwrotności i przechodniości porządku, co można prosto wykazać.

\begin{proof}
    Załóżmy, że relacja $\prec$ na zbiorze $A$ jest przeciwzwrotna i przechodnia. Weżmy $a, b\in A$,
    wówczas z przechodniości
    \begin{equation*}
        a \prec b \land b \prec a \imp a \prec a \ifff \lnot(a \prec b \land b \prec a) \lor a \prec a.
    \end{equation*}
    Jednak z przeciwzwrotności $\prec$ wiemy, iż zdanie $a\prec a$ jest fałszywe dla dowolnego $a$ ze
    zbioru $A$, dlatego też
    \begin{equation*}
        \lnot(a \prec b \land b \prec a) \lor a \prec a \imp \lnot(a \prec b \land b \prec a)\ifff
        \lnot (a \prec b) \lor \lnot (b \prec a) \ifff (a\prec b\imp \lnot(b\prec a)).
    \end{equation*}
\end{proof}

\begin{defr}{Definicja 1.6: Liniowy porządek}
    Porządek częściowy $\preceq$ (lub $\prec$) na zbiorze $A$ nazywamy \textbf{porządkiem liniowym (pełnym)}
    na zbiorze $A$, jeżeli jest \textbf{spójny}.
\end{defr}

Spójność dla porządku ostrego formułujemy następująco: $(\forall a,b\in A)(a\ne b\imp a\prec b \lor b\prec a)$.

\begin{defr}{Definicja 1.7: Elementy wyróżnione}
    Dany jest zbiór $A$ z porządkiem częściowym $\preceq$. Niech $B\subset A$ i $c\in A$. Mówimy, że:
    \begin{enumerate}[label=\Roman*.]
        \item $c$ jest \textbf{ograniczeniem górnym} zbioru $B$, jeśli $(\forall b\in B)\ b \preceq c.$
        \item $c$ jest \textbf{ograniczeniem dolnym} zbioru $B$, jeśli $(\forall b\in B)\ c \preceq b.$
        \item $c$  jest \textbf{kresem górnym} (ozn. $\sup{A}$) zbioru B, jeśli:
            \begin{enumerate}
            \item jest ograniczeniem górnym.
            \item dla dowolnie innego ograniczenia górnego $c'$ zbioru $B$ zachodzi $c\preceq c'$.
            \end{enumerate}
        \item $c$  jest \textbf{kresem dolnym} (ozn. $\inf{A}$) zbioru B, jeśli:
            \begin{enumerate}
            \item jest ograniczeniem dolnym.
            \item dla dowolnie innego ograniczenia dolnego $c'$ zbioru $B$ zachodzi $c'\preceq c$.
            \end{enumerate}
        \item $c$ jest elementem \textbf{maksymalnym} zbioru $B$, jeśli $c\in B \land
        \lnot(\exists b\in B)\
        c\prec b$.
        \item $c$ jest elementem \textbf{największym} zbioru $B$, jeśli $c\in B \land
        (\forall b\in B)\ b 
        \preceq c$.
        \item $c$ jest elementem \textbf{minimalnym} zbioru $B$, jeśli $c\in B \land
        \lnot(\exists b\in B)\ 
        b\prec c$.
        \item $c$ jest elementem \textbf{najmniejszym} zbioru $B$, jeśli $c\in B \land
        (\forall b\in B)\ 
        c\preceq b$.
    \end{enumerate}
\end{defr}

Powyższe pojęcia\footnote{Na wykładzie pojawiły się wszystkie wymienione terminy, z
wyjątkiem elementu najmniejszego i największego. Zapewne dlatego, że dla porządku liniowego, który
został przyjęty, nie ma rozróżnienia między elementem największym a maksymalnym.}
na wykładzie zostały zdefiniowane tylko dla liniowo uporządkowanego zbioru $A$,
ale można je bez problemu uogólnić na zbiór z porządkiem częściowym, co też zrobiłem. Warto dodać, iż dla porządków
liniowych element największy i maksymalny znaczą to samo. Analogicznie jest z elementem najmniejszym
i minimalnym. Sprawy mają się inaczej w przypadku porządków częściowych. Oczywiście, element
największy jest również i maksymalny. Jednak implikacja w drugą stronę już nie zawsze zachodzi.
Obrazem tego stanu rzeczy są podane \href{https://calcworkshop.com/wp-content/uploads/hasse-diagram-find-the-maximal-minimal-least-and-greatest.png}{{\color{blue} \underline{diagramy Hassego}}}.

\begin{twier}{Twierdzenie 1.2}
    Dane są dwie relacje $\preceq$ i $\prec$ w zbiorze $A$. Jeśli spełniają one następujące warunki:
    \begin{enumerate}[label=(\alph*)]
        \item $(\forall a,b\in A)(a\preceq b \ifff a\prec b \lor a=b)$
        \item $(\forall a,b\in A)(a\prec b \ifff a\preceq b \land a\ne b)$,
    \end{enumerate}
    wówczas $\preceq$ jest porządkiem słabym, wtedy i tylko wtedy gdy $\prec$ jest porządkiem ostrym. 
\end{twier}

\begin{defr}{Definicja 1.8: Relacja odwrotna}
    Niech $R\subset A\times B$. \textbf{Relacją odwrotną} $R^{-1}$ do relacji $R$ nazywamy zbiór
    \begin{equation*}
        R^{-1}:=\set{\op{a,b}\in A\times B:\op{a,b}\in R}.
    \end{equation*}
    Innymi słowy $(\forall a\in A)(\forall b\in B)(bR^{-1}a\ifff aRb)$.
\end{defr}

\newpage


%%%%%%%%%%%%%%%%%%%%%%%%%%%%%%%%%%%%%%%%%%%%%%%%%%%%%%%%%%%%%%%%%%%%%%%%%%%%%%%%%%%%%%%%%%%%%%%%%%%%%%%%%%%%%%%%
                                \section*{Sekcja 2} \smallskip
                                {\Huge\bfseries Funkcje} \bigskip \medskip
%%%%%%%%%%%%%%%%%%%%%%%%%%%%%%%%%%%%%%%%%%%%%%%%%%%%%%%%%%%%%%%%%%%%%%%%%%%%%%%%%%%%%%%%%%%%%%%%%%%%%%%%%%%%%%%%

\begin{defr}{Definicja 2.1: Funkcja}
    Relację $f$ między elementami zbioru $A$ i elementami zbioru $B$ nazywamy \textbf{funkcją}, jeżeli
    \begin{equation*}
        (\forall x\in A)(\exists! y\in B)\ \op{x,y}\in f.
    \end{equation*}
    Powyższe zdanie można zapisać równoważnie jako
    \begin{equation*}
        (\forall x\in A)(\exists y\in B)\ \op{x,y}\in f\ \land\ ((\forall x\in A)(\forall y_1,y_2\in B)
        (\op{x,y_1}\in f\land \op{x,y_2}\in f\imp y_1=y_2)).
    \end{equation*}
    \begin{itemize}
        \item \textbf{Dziedziną} (ozn. $dom(f)$ lub $D_f$) funkcji $f$ nazywamy zbiór $A$.
        \item \textbf{Przeciwdziedziną} (ozn. $\rotatebox[origin=c]{180}{$D$}_f$)
        funkcji $f$ nazywamy zbiór $B$.
        \item \textbf{Zbiorem wartości} (ozn. $rng(f)$ lub $R_f$) funkcji $f$ nazywamy zbiór \\
        $R_f = \set{y\in B: (\exists x\in A)\ \op{x,y}\in f}\subset B$.
    \end{itemize}
\end{defr}

\begin{defr}{Definicja 2.2: Obraz i przeciwobraz}
    Niech $f:A\to B$ oraz $C\subset A$ i $D\subset B$.
    \begin{enumerate}[label=(\arabic*)]
        \item \textbf{Obrazem} zbioru $C$ względem funkcji $f$ nazywamy zbiór
        \begin{equation*}
            f[C]=\set{y\in B: (\exists x\in C)\ y=f(x)}=\set{f(x):x\in C}.
        \end{equation*}
        \item \textbf{Przeciwobrazem} zbioru $D$ względem funkcji $f$ nazywamy zbiór
        \begin{equation*}
            f^{-1}[D]=\set{x\in A:f(x)\in D}.
        \end{equation*}
    \end{enumerate}
\end{defr}

Supremum funkcji $f$ na zbiorze $C$ $\sup_{x\in C}f(x)$ jest kresem górnym obrazu zbioru $C$
względem niej. Analogicznie definiujemy $\inf_{x\in C}f(x)$, $\max_{x\in C}f(x)$ i
$\min_{x\in C}f(x)$.

\begin{defr}{Definicja 2.3: Injekcja}
    Relację funkcyjną $f\subset A\times B$ nazywamy \textbf{injekcją} (różnowartościową), jeżeli
    \begin{gather*}
        (\forall x_1,x_2\in A)(f(x_1)=f(x_2)\imp x_1=x_2)\text{, czyli równoważnie}\\
        (\forall x_1,x_2\in A)(\forall y\in B)(\op{x_1,y}\in f \land \op{x_2,y}\in f\imp x_1=x_2).
    \end{gather*}
\end{defr}

\begin{defr}{Definicja 2.4: Surjekcja}
    Mówimy, że relacja funkcyjna $f\subset A\times B$ jest ze zbioru $A$ \textbf{na} zbiór $B$, jeśli
    \begin{equation*}
        (\forall y\in B)(\exists x\in A)\ y=f(x)\text{, czyli }(\forall y\in B)(\exists x\in A)\ 
        \op{x,y}\in f.
    \end{equation*}
    Funkcję taką nazywamy też \textbf{surjekcją}.
\end{defr}

\begin{defr}{Definicja 2.5: Bijekcja}
    Relację funkcyjną, która jest zarówno injekcją jak i surjekcją nazywamy \textbf{bijekcją}.
\end{defr}

\begin{defr}{Definicja 2.6: Funkcja odwrotna}
    Jeśli $f:A\rightarrow B$ jest bijekcją, to \textbf{funkcją odwrotną} do $f$
    jest funkcja $f^{-1}:B\rightarrow A$, taka że
    \begin{equation*}
        (\forall x\in A)(\forall y\in B)(\op{y,x}\in f^{-1}\ifff \op{x,y}\in f)
    \end{equation*}
\end{defr}

\begin{twier}{Twierdzenie 2.1}
    \begin{enumerate}[label=(\arabic*), nosep]
        \item Jeżeli funkcja jest bijekcją, to posiada funkcję odwrotną, 
        która również jest bijekcją
        \item Jeżeli funkcja jest odwracalna, to oznacza, że jest bijekcją.
    \end{enumerate}
\end{twier}

\begin{defr}{Definicja 2.7: Złożenie Funkcji}
    \hfill \texttt{WERSJA I}\\ \smallskip
    Niech $f:A\to B$, $g:B\to C$ (wystarczy nawet założyć, że
    $g:B_1\to C$, jesli $B\subset B_1$)
    i $x\in A$. \textbf{Złożeniem funkcji} $f$ z funkcją $g$ nazywamy
    funkcję $g\circ f:A\to C$, określoną wzorem $(g\circ f)(x)=g(f(x))$.
    \\ \medskip
    \hfill \texttt{WERSJA II}\\ \smallskip
    \textbf{Złożeniem funkcji} $f$ i $g$ nazywamy funkcję $g\circ f$
    zdefiniowaną wzorem
    \begin{equation*}
        g\circ f=\set{\op{x,z}\in D_f\times R_g:\exists y[\op{x,y}\in f
        \land \op{y,z}\in g]}.
    \end{equation*}
\end{defr}

Wyrażenie $(g\circ f)(x)=g(f(x))$ można zapisać alternatywnie jako:
\begin{equation*}
    (\forall x\in A)(\forall z\in C)\ \op{x,z}\in g\circ f \ifff (\exists y\in B)
    (\op{x,y}\in f\land \op{y,z}\in g).
\end{equation*}

Zauważmy różnicę między tymi dwiema wersjami
\footnote{Tylko wersja pierwsza pojawiła się na wykładzie. Ta druga jest
tylko moim dodatkiem. Podaję ją tutaj, bo chociaż nie pojawiła się w czasie
wykładu, to posługiwaliśmy się jej wyróżniającą własnością na ćwiczeniach.}
definicji złożenia funkcji.
Pierwsza, częściej spotykana, zakłada że $R_f\subset D_g$, skąd wynika, że
$D_{g\circ f}=D_f$. Natomiast według drugiej definicji, złożenie $g\circ f$
funkcji $f$ i $g$ ma następujące własności.
\begin{enumerate}[label=(\alph*)]
    \item $D_{g\circ f}=\set{x\in D_f:f(x)\in D_g}$,
    \item $(\forall x\in D_{g\circ f})\ (g\circ f)(x)=g(f(x))$.
\end{enumerate} 

\begin{twier}{Twierdzenie 2.2}
    Dla dowolnych funkcji $f$, $g$, $h$ zachodzi równość
     $(f\circ g)\circ h= f\circ (g\circ h)$.
\end{twier}

\begin{defr}{Definicja 2.8: Funkcja identycznościowa}
    Dla dowolnego niepustego zbioru $A$ możemy określić \textbf{funkcję identycznościową}
    na zbiorze $A$ (identyczność na zbiorze $A$) następująco:
    \begin{equation*}
        id_A:A\to A,\quad (\forall x\in A)\ id_A(x)=x.
    \end{equation*}
\end{defr}

\begin{twier}{Twierdzenie 2.3}
    Jeśli $f:A\to B$ i $f^{-1}:B\to A$, to $f^{-1}\circ f:A\to A$ jest identycznością na
    zbiorze $A$.
\end{twier}

\begin{defr}{Definicja 2.9: Obcięcie i przedłużenie funkcji}
    Niech $f:A\to B$.
    \begin{enumerate}[label=(\arabic*)]
        \item Niech $C\subset X$. \textbf{Obcięciem funkcji} $f$ do zbioru
        $C$ nazywamy funkcję $f|_C:C\to B$,  $(f|_C)(x)=f(x)$.
        \item Funkcję $g:C\to B$ nazywamy \textbf{przedłużeniem funkcji}
        $f$, jeśli $A\subset C$ oraz $(\forall x\in A)\ f(x)=g(x)$.
    \end{enumerate}
\end{defr}

Zwróćmy uwagę, że daną funkcję $f$ można przedłużyć na dany właściwy nadzbiór
jej dziedziny na różne sposoby. Zauważmy też, że funkcja jest zawsze przedłużeniem
swojego obcięcia.

\newpage


%%%%%%%%%%%%%%%%%%%%%%%%%%%%%%%%%%%%%%%%%%%%%%%%%%%%%%%%%%%%%%%%%%%%%%%%%%%%%%%%%%%%%%%%%%%%%%%%%%%%%%%%%%%%%%%%
                                \section*{Sekcja 3}\smallskip
                                {\Huge\bfseries Równoliczność} \bigskip \medskip
%%%%%%%%%%%%%%%%%%%%%%%%%%%%%%%%%%%%%%%%%%%%%%%%%%%%%%%%%%%%%%%%%%%%%%%%%%%%%%%%%%%%%%%%%%%%%%%%%%%%%%%%%%%%%%%%

\begin{defr}{Definicja 3.1: Równoliczność}
    Mówimy, że zbiory $A$ i $B$ są \textbf{równoliczne} (ozn. $|A|=|B|$, $A\sim B$),
    gdy istnieje bijekcja $f:A\to B$.
\end{defr}

Równoliczność \textit{ma własności relacji równoważności} 
(jest zwrotna, symetryczna i przechodnia)
i faktycznie nią jest, gdy ograniczymy relację równoliczności do zbioru $\Po(U)$
\footnote{$\Po(U)$ to zbiór potęgowy pewnego zbioru $U$, czyli zbiór wszystkich
podzbiorów $U$. Równoliczność ograniczamy do jakiegoś
zbioru potęgowego, bo jej dziedzina i obraz nie są normalnie zbiorami,
 więc nie byłaby ona relacją równoważności w ścisłym sensie.}. 
 Jeśli $A, B\in \Po(U)$ i $R$ będzie symbolizować relację równoliczności, 
 to możemy przyjąć, iż $|A|=|B|$ oznacza, że
$[A]_R=[B]_R$.

\begin{defr}{Definicja 3.2: Zbiór skończony i nieskończony}
    O zbiorze $A$ mówimy, że jest \textbf{skończony}, jeżeli jest pusty lub równoliczny
    jakiemuś zbiorowi postaci $\set{1,\ldots,n}_{n\in\N}$. Piszemy wówczas, że $|A|=n$.
    Zbiór, który nie jest skończony nazywamy \textbf{nieskończonym}.
\end{defr}

\begin{defr}{Definicja 3.3: Zbiór przeliczalny}
    Mówimy, że zbiór $A$ jest \textbf{przeliczalny}, jeżeli jest równoliczny zbiorowi
    $\N$. Piszemy wówczas, że $|A|=\aleph_0$. 
\end{defr}

Zbiór nazywamy \textit{co najwyżej przeliczalnym}, jeśli jest on skończony lub przeliczalny.

\begin{defr}{Definicja 3.4: Zbiór nieprzeliczalny}
    Mówimy, że zbiór $A$ jest \textbf{nieprzeliczalny}, jeżeli nie jest przeliczalny,
    ani skończony.
\end{defr}

Zbiór liczb rzeczywistych jest zbiorem nieprzeliczalnym. |$\R$| oznaczamy 
jako $\mathfrak{c}$ lub $2^{\aleph_0}$ i nazywamy \textit{continuum}. Continuum jest
większe od mocy $\N$.

\newpage


%%%%%%%%%%%%%%%%%%%%%%%%%%%%%%%%%%%%%%%%%%%%%%%%%%%%%%%%%%%%%%%%%%%%%%%%%%%%%%%%%%%%%%%%%%%%%%%%%%%%%%%%%%%%%%%%
                                \section*{Sekcja 4} \smallskip
                                {\Huge\bfseries Liczby rzeczywiste} \bigskip \medskip
%%%%%%%%%%%%%%%%%%%%%%%%%%%%%%%%%%%%%%%%%%%%%%%%%%%%%%%%%%%%%%%%%%%%%%%%%%%%%%%%%%%%%%%%%%%%%%%%%%%%%%%%%%%%%%%%

Dany jest zbiór liczb rzeczywistych \(\R\) z określonymi działaniami $+$ i $\cdot$
oraz porządkiem liniowym $\le$.
Aksjomaty teorii liczb rzeczywistych podzielimy na
trzy kategorie: aksjomaty ciała przemiennego, aksjomaty porządku, oraz aksjomat
ciągłości\footnote{Aksjomaty przepisałem ze skryptu Strzeleckiego, ponieważ
były tam zapisane w trochę bardziej eleganckiej postaci.}.

\subsection*{{\color{darkred} Aksjomaty ciała przemiennego}} 
\begin{enumerate}[label=(\arabic*)]
    \item \textbf{Przemienność dodawania}
    \((\forall a,b\in\R)\ a+b=b+a\).
    \item \textbf{Łączność dodawania}
    \((\forall a,b,c\in\R)\ a+(b+c)=(a+b)+c\).
    \item \textbf{Charakteryzacja zera}
    \((\exists 0\in\R)(\forall a\in\R)\ a+0=a\).
    \item \textbf{Istnienie elementów przeciwnych}
    \((\forall a\in\R)(\exists -a\in\R)\ a+(-a)=0\).
    \item \textbf{Przemienność mnożenia}
    \((\forall a,b\in\R)\ a\cdot b=b\cdot a\).
    \item  \textbf{Łączność mnożenia}
    \((\forall a,b,c\in\R)\ a\cdot(b\cdot c)=(a\cdot b)\cdot c\).
    \item \textbf{Charakteryzacja jedynki}
    \((\exists 1\in\R)(\forall a\in\R)\ a\cdot 1=a\).
    \item \textbf{Istnienie elementów odwrotnych}
    \((\forall a\in\R)(\exists a^{-1}\in\R)\ a\cdot a^{-1}=1\).
    \item \textbf{Rozdzielność mnożenia względem dodawania}
    \((\forall a,b,c\in\R)\ a\cdot(b+c)=a\cdot b + a\cdot c\).
\end{enumerate}

\subsection*{{\color{darkred} Aksjomaty porządku}} 
\begin{enumerate}[label=(\arabic*)]
    \item \textbf{Prawo trichotomii}
    \footnote{Aksjomat ten uwzględniamy, gdy przyjmujemy $(\R,+,\cdot,<)$,
    na wykładzie natomiast przyjęliśmy $(\R,+,\cdot,\le)$.}
    \((\forall a,b\in\R)\) zachodzi \emph{dokładnie jedna} z trzech możliwości:
    \begin{equation*}
        a<b,\quad a=b,\quad b<a.
    \end{equation*}
    \item \textbf{Przechodniość}
    \((\forall a,b,c\in\R)(a\le b\land b\le c\imp a\le c)\).
    \item \textbf{Związki nierówności z działaniami}
    \begin{enumerate}
        \item \((\forall a,b,c\in\R)(a\le b\imp a+c\le b+c)\);
        \item \((\forall a,b,c\in\R)(a\le b\land 0\le c\imp ac\le bc)\).
    \end{enumerate}
\end{enumerate} \smallskip

\begin{dedekind}
    \textit{Każdy niepusty, ograniczony z góry podzbiór} $A\subset\R$
    \textit{ma kres górny} $M=\sup{A}\in\R$.
\end{dedekind} \pagebreak

\begin{twier}{Twierdzenie 4.1: Aksjomat Archimedesa}
    Dla każdej pary dodatnich liczb rzeczywistych $a$ i $b$ istnieje liczba
    naturalna $n$, taka że $a<nb$.
\begin{equation*}
    (\forall a,b\in\R_+)(\exists n\in\N)\ a<nb
\end{equation*}
\end{twier}

Twierdzenie to, choć bywa tak zwyczajowo nazywane,
na prawdę aksjomatem w arytmetyce nie jest,
bo wynika z innych aksjomatów teorii liczb rzeczywistych. 

\begin{defr}{Definicja 4.1: Przekrój Dedekinda}
Podział zbioru liczb wymiernych na parę zbiorów $\op{A,B}$, spełniające
warunki\footnote{Podane warunki różnią się swoją postacią,
tym co były przedstawione na wykładzie, niemniej jednak są im równoważne.}:
\begin{enumerate}[label=(\arabic*)]
    \item $A\neq \emptyset \land B\neq \emptyset$,
    \item $A\cup B=\Q$,
    \item $A\cap B=\emptyset$
    \item $(\forall a\in A)(\forall b\in B)\ a<b$.
\end{enumerate}
nazywamy \textbf{przekrojem Dedekinda} zbioru $\Q$. Zbiór $A$
nazywany jest \textbf{klasą dolną} przekroju, a zbiór $B$
\textbf{klasą górną}.
\end{defr}

Przekrój Dedekinda $\op{A,B}$ zdefiniowany w taki sposób może mieć jedną
z trzech następujących postaci, w której: \smallskip
\begin{enumerate}[nosep]
    \item w zbiorze $A$ istnieje element największy,
    \item w zbiorze $B$ istnieje element najmniejszy,
    \item w zbiorze $A$ nie istnieje element największy i
    w zbiorze $B$ nie istnieje element najmniejszy.
\end{enumerate} \smallskip
W trzecim przypadku przekrój wyznacza tzw. \textit{lukę}. 
Aksjomat ciągłości w ujęciu przekrojowym, mówi o tym, że żaden
z przekrojów Dedekinda zbioru $\R$ nie wyznacza luki.\smallskip

Przekroje typu 1 i 2 nazywamy \textit{liczbami rzeczywistymi wymiernymi}.
Dwa przekroje typu 1 i 2 mogą wyznaczać tę samą liczbę wymierną. Relację
równoważności przekrojów zdefiniujemy poniżej. Natomiast przekrój
$\op{A,B}$ wyznaczający lukę nazywamy \textit{liczbą rzeczywistą
niewymierną}. \bigskip

Zdefiniujmy relację równoważności $R$ przekrojów Dedekinda:
\begin{equation*}
    \op{A_1,B_2}\ R\ \op{A_2,B_2}\ifff A_1=A_2\ \lor\ \exists\max{A_1},
    \min{B_2}(\max{A_1}=\min{B_2})\ \lor\ \exists\max{A_2},\min{B_1}
    (\max{A_2}=\min{B_1}).
\end{equation*}



Nasze rozważania doprowadzają nas do \textit{\textbf{{\color{darkred} konstrukcji
zbioru liczb rzeczywistych za pomocą przekrojów Dedekinda}}}\footnote{
W tym miejscu notatki z liczb rzeczywistych na razie zakańczam. Dalsze
wyprowadzenia operacji na liczbach rzeczywistych odkładam na czas bliższy
terminowi egzaminu ustnego. Na kolosie zagadnienia te raczej nie będą potrzebne.
}.
\begin{equation*}
    \R:=\ilorazowy{\set{\op{A,B}:\text{przekroje Dedekinda}}}
\end{equation*}
\newpage


%%%%%%%%%%%%%%%%%%%%%%%%%%%%%%%%%%%%%%%%%%%%%%%%%%%%%%%%%%%%%%%%%%%%%%%%%%%%%%%%%%%%%%%%%%%%%%%%%%%%%%%%%%%%%%%%
                                \section*{Sekcja 5} \smallskip
                                {\Huge\bfseries Ciągi} \bigskip \medskip
%%%%%%%%%%%%%%%%%%%%%%%%%%%%%%%%%%%%%%%%%%%%%%%%%%%%%%%%%%%%%%%%%%%%%%%%%%%%%%%%%%%%%%%%%%%%%%%%%%%%%%%%%%%%%%%%

\begin{defr}{Definicja 5.1: Ciąg nieskończony}
    \textbf{Ciągiem} (nieskończonym) o elementach w zbiorze \(A\)
    nazywamy dowolną funkcję $a:\N\to A$ (ozn. $a_n$, $\set{a_n}_{n\in\N}$
    i $(a_n)_{n\in\N}$).
\end{defr}

\begin{defr}{Definicja 5.2: Podciąg}
    Jeżeli $\seq{a}$ jest ciągiem o elementach w zbiorze $A$ oraz
    $\seq{k}$ jest ciągiem ściśle rosnącym o elementach w $\N$,
    to ciąg $a\circ k=\set{a_{k_n}}_{n\in\N}$ nazywamy \textbf{podciągiem}
    ciągu $\seq{a}$.
\end{defr}

\begin{defr}{Definicja 5.3: Granica ciągu}
    Mówimy, że $g\in \R$ jest granicą ciągu $\seq{a}$, jeżeli
    \begin{equation*}
        \forall_{\varepsilon>0}\ \exists_{n_\varepsilon\in\N}\ 
        \forall_{n\ge n_\varepsilon, n\in\N}\hquad
        |a_n-g|<\varepsilon.
    \end{equation*}
\end{defr}

Gdy $g$ jest granicą ciągu $\seq{a}$, to piszemy, że $\lin{a_n}=g$
lub $a_n\arn g$.

\begin{twier}{Twierdzenie 5.1}
    Ciąg może mieć co najwyżej jedną granicę.
\end{twier}

\begin{defr}{Definicja 5.4: Ciąg Cauchy'ego}
    \textbf{Ciągiem Cauchy'ego} nazywamy ciąg $\seq{a}$, jeśli
    \begin{equation*}
        \forall_{\varepsilon>0}\ \exists_{M\in\N}\
        \forall_{n,m\geq M}\ \forall_{m,n\in\N}\hquad
        |a_n-a_m|<\varepsilon.
    \end{equation*}
\end{defr}

\begin{twier}{Twierdzenie 5.2}
    Każdy ciąg liczb rzeczywistych jest zbieżny, wtedy i tylko wtedy gdy spełnia warunek Cauchy'ego (tzn. jest ciągiem Cauchy'ego).
\end{twier}

\begin{twier}{Twierdzenie 5.3}
    Każdy ciąg zbieżny jest ograniczony, zarówno z dołu, jak i z góry.
\end{twier}

\begin{twier}{Twierdzenie 5.4}
    Jeżeli ciąg $\seq{a}$ jest rozbieżny do $+\infty$ ($-\infty$),
    to jest ograniczony z dołu (góry).
\end{twier}

\begin{twier}{Twierdzenie 5.5}
Jeśli ciąg $\seq{a}$ jest zbieżny do $g$ to każdy podciąg ciągu $\set{a_n}$ też
jest zbieżny do $g$.
\end{twier}

\begin{twier}{Twierdzenia o „arytmetyce” granic 5.6}
    Niech $\seq{a}$ i $\seq{b}$ będą ciągami liczbowymi oraz $a,b \in\R$.
    \begin{enumerate}[label=(\arabic*)]
        \item $a_n\arn a \implies |a_n|\arn|a|$.
        \item $a_n\arn a \implies \sqrt{a_n}\arn\sqrt{a}$.
        \item $a_n\arn a\ \land\  b_n\arn b \implies a_n+b_n\arn a+b$.
        \item $a_n\arn a\ \land\ b_n\arn b\implies a_n-b_n\arn a-b$.
        \item $a_n\arn+\infty\ \land\ \left[b_n\right. \implies a_n+b_n\arn +\infty$.
        \item $a_n\arn-\infty\ \land\ \left.b_n\right] \implies a_n+b_n\arn -\infty$.
        \item $a_n\arn a\ \land\ b_n\arn b\implies a_n b_n \arn ab$.
        \item $a_n\arn 0\ \land\ [b_n]\implies a_nb_n\arn 0$.
        \item $a_n\arn+\infty\ \land\ (\exists q>0)\hquad dddn\hquad b_n>q\implies a_nb_n\arn+\infty$.
        \item $a_n\arn+\infty\ \land\ (\exists q<0)\hquad dddn\hquad b_n<q\implies a_nb_n\arn-\infty$.
        \item $a_n\arn a\ \land\ b_n\arn b\neq0\ \land\ (\forall n\in\N)\ b_n\neq0\implies\frac{a_n}{b_n}\arn\frac{a}{b}$.
        \item $[a_n]\ \land\ |b_n|\arn+\infty\implies \frac{a_n}{b_n}\arn0$.
        \item $a_n\arn0\ \land\ (\exists c>0)\hquad dddn\hquad |b_n|>c \implies \frac{a_n}{b_n}\arn0$.
        \item $a_n\arn+\infty\ \land\ [b_n]\ \land\ dddn\hquad b_n>0\implies \frac{a_n}{b_n}\arn+\infty$.
        \item $a_n\arn-\infty\ \land\ [b_n]\ \land\ dddn\hquad b_n>0\implies \frac{a_n}{b_n}\arn-\infty$.
        \item $a_n\arn+\infty\ \land\ [b_n]\ \land\ dddn\hquad b_n<0\implies \frac{a_n}{b_n}\arn-\infty$.
        \item $a_n\arn-\infty\ \land\ [b_n]\ \land\ dddn\hquad b_n<0\implies \frac{a_n}{b_n}\arn+\infty$.
        \item $a_n\arn a\ \land\ b_n\arn b\implies a_n^{b_n}\arn a^b$.
        \item $a_n\arn a>1\ \land\ b_n\arn +\infty\implies a_n^{b_n}\arn +\infty$.
        \item $a_n\arn a\in(0,1)\ \land\ b_n\arn +\infty\implies a_n^{b_n}\arn 0$.
        \item $a_n\arn a>1\ \land\ b_n\arn -\infty\implies a_n^{b_n}\arn 0$.
        \item $a_n\arn a\in(0,1)\ \land\ b_n\arn -\infty\implies a_n^{b_n}\arn +\infty$.
        \item $a_n\arn +\infty\ \land\ b_n\arn b>0\implies a_n^{b_n}\arn +\infty$.
        \item $a_n\arn +\infty\ \land\ b_n\arn b<0\implies a_n^{b_n}\arn 0$.
    \end{enumerate}
\end{twier}

W celu uproszeczenia zapisu, wprowadziłem do powyższych twierdzeń parę (autorskich!) oznaczeń.
\begin{align*}
    [a_n\quad &ozn. &(\exists m\in\R)(\forall n\in\N)\ &a_n\ge m\\
    a_n]\quad &ozn. &(\exists M\in\R)(\forall n\in\N)\ &a_n\le M\\
    [a_n]\quad &ozn. &(\exists M\in\R)(\forall n\in\N)\ &|a_n|\le M\\
    %)a_n\quad &ozn. &(\forall m\in\R)(\exists N\in\N)\ &a_N<m\\
    %a_n(\quad &ozn. &(\forall M\in\R)(\exists N\in\N)\ &a_N>M\\
    %)a_n(\quad &ozn. &(\forall M\in\R)(\exists N\in\N)\ &|a_N|>M\\
    %[a_n(\quad &ozn. &&[a_n\ \land\ a_n(\\
    %&itd.
\end{align*}

Powyższe twierdzenia, jak i jakiekolwiek inne twierdzenia o arytmetycznych własnościach
granic nie mają zastosowania w przypadku tzw. \textit{wyrażeń nieoznaczonych}.
Do opisania takich wyrażeń wykorzystuje się następujące symbole:
\begin{equation*}
\infty -\infty  ,\hquad 0\cdot\infty,\hquad \frac{0}{0},\hquad \frac{\infty}{\infty},
\hquad 1^{\infty},\hquad \infty^{0},\hquad 0^0.  
\end{equation*}

\begin{twier}{Twierdzenie 5.7}
    Jeżeli ciąg $\seq{a}$ jest zbieżny, a $\seq{b}$ nie, to
    ciąg $\seq{c}$ postaci $c_n=a_n+b_n$ również \underline{nie} jest zbieżny.
\end{twier}

Przydatną własnością wynikającą z tw. 5.7 i tw. 5.6 jest to, że jeśli ciąg $b_n=a_n-g\arn 0$, to $(a_n)$ jest zbieżny i to
dokładnie do granicy $g$.

\begin{twier}{Twierdzenie o szacowaniu granic 5.8}
Załózmy, że $\seq{a}$ i $\seq{b}$ są zbieżnymi ciągami liczb rzeczywistych
oraz $x\in\R$. Zachodzą wówczas następujące implikacje:
\begin{enumerate}[label=(\roman*)]
    \item $\lim\limits_{n\to\infty}{a_n}>x\implies dddn\hquad a_n>x$,
    \item $\lim\limits_{n\to\infty}{a_n}<x\implies dddn\hquad a_n<x$,
    \item $\lim\limits_{n\to\infty}{a_n}>\lim\limits_{n\to\infty}{b_n}\implies dddn\hquad a_n>b_n$,
    \item $dddn\hquad a_n\le b_n\implies \lim\limits_{n\to\infty}{a_n}\le\lim\limits_{n\to\infty}{b_n}$.
\end{enumerate}
\end{twier}

Na wykładzie przedstawiono tylko (iv). (iii) jest równoważne (iv) z prawa kontrapozycji. (i) da się łatwo wywieść z (iii),
wystarczy bowiem przyjąć, że $b_n$ jest ciągiem stałym stale równym $x$. (ii) dowodzimy analogicznie.

\begin{twier}{Twierdzenie o dwóch ciągach 5.9}
Niech $\seq{a}$ i $\seq{b}$ będą ciągami liczb rzeczywistych, wówczas zachodzą następujące implikacje:
\begin{enumerate}[label=\Roman*.]
    \item $\lin{a_n}=+\infty\ \land\ dddn\hquad b_n\ge a_n\implies \lin{b_n}=+\infty$,
    \item $\lin{a_n}=-\infty\ \land\ dddn\hquad b_n\le a_n\implies \lin{b_n}=-\infty$.
\end{enumerate}
\end{twier}

\begin{twier}{Twierdzenie o trzech ciągach 5.10}
Jeżeli $\seq{a}$, $\seq{b}$, $\seq{c}$ są ciągami liczb rzeczywistych, to wówczas
\begin{equation*}
    \lin{a_n}=\lin{c_n}=g\ \land\ dddn\hquad a_n\le b_n\le c_n\implies \lin{b_n}=g.
\end{equation*}
\end{twier}

\begin{twier}{Kryterium d'Alamberta dla ciągów 5.11}
Niech $\seq{a}$ i $\seq{b}$ będą ciągami liczb rzeczywistych, wówczas zachodzą następujące implikacje:
\begin{enumerate}[label=\Roman*.]
    \item $dddn\hquad a_n>0\ \land\ (\exists q>1)\hquad dddn\hquad \frac{a_{n+1}}{a_n}>q\implies \lin{a_n}=+\infty$,
    \item $dddn\hquad a_n>0\ \land\ (\exists q<1)\hquad dddn\hquad \frac{a_{n+1}}{a_n}<q\implies \lin{a_n}=0$.
\end{enumerate}
\end{twier}

\smallskip

\begin{twier}{Twierdzenie o ciągu monotonicznym i ograniczonym 5.12}
\begin{enumerate}[label=\Roman*.]
    \item Każdy niemalejący i ograniczony z góry ciąg $\seq{a}$ jest zbieżny do
    \[\sup\limits_{n\in\N}{a_n}=\sup{\set{a_n:n\in\N}}.\]
    \item Każdy nierosnący i ograniczony z dołu ciąg $\seq{a}$ jest zbieżny do
    \[\inf\limits_{n\in\N}{a_n}=\inf{\set{a_n:n\in\N}}.\]
\end{enumerate}
\end{twier}

\begin{twier}{Twierdzenie Bolzano-Weierstrassa 5.13}
Jeżeli ciąg $\seq{a}\subset \R$ jest ograniczony (zarówno z góry jak i z dołu), to posiada podciąg zbieżny.
\end{twier}

\begin{defr}{Definicja 5.5: Liczba Eulera}
    \textbf{Liczbą Eulera} nazywamy niewymierną liczbę, zdefiniowaną jako granicę 
    \begin{equation*}
        e=\lin{\Bigl(1+\frac{1}{n}\Bigl)^n}.
    \end{equation*}
\end{defr}

\begin{twier}{Twierdzenie 5.14}
Ciąg $e_n=(1+ 1/n)^n$ jest niemalejący, a $\tilde{e}_n=(1+1/n)^{n+1}$ nierosnący. Ponadto
\begin{equation*}
    (\forall n\in\N_+)\hquad \Bigl(1+\frac{1}{n}\Bigl)^n < e < \Bigl(1+\frac{1}{n}\Bigl)^{n+1}.
\end{equation*}
\end{twier}

\textit{Parę przykładów granic ciągów:}
\begin{align}
    \lin{\frac{q^n}{n^k}}=\ &
        \begin{cases}
            +\infty, &\text{dla }\ q>1 \lor(q=1\land k<0)\\
            1, &\text{dla }\ q=1\land k=0\\
            0, &\text{dla }\ q<1\lor(q=1\land k>0)
        \end{cases}\text{, gdzie }q\ge0.\\
    \lin{\sqrt[n]{n}}=\ &1\\
    \lin{\Bigl(1+\frac{1}{n} \Bigl)^{n+1}}=\ &e\\
    \lin{\sum_{k=0}^{n}\frac{1}{k!}}=\ &e
\end{align}\smallskip

\begin{twier}{Twierdzenie 5.15}
    Jeżeli $\seq{a}$ i $\seq{b}$ są ciągami liczb rzeczywistych, to wówczas
    \begin{equation*}
        \lin{a_n}=0\ \land\ \lin{a_n b_n}=g\implies \lin{(1+a_n)^{b_n}}=e^g.
    \end{equation*}
\end{twier}

\begin{twier}{Uogólniona nierówność Bernoulliego 5.16}
    Klasyczną nierówność Bernoulliego można uogólnić do poniższej postaci.
\begin{align*}
    (1+x)^r\ge1+rx,\quad&\text{dla}\hquad x>-1\ \land\ r\ge1\\\
    (1+x)^r\le1+rx,\quad&\text{dla}\hquad x>-1\ \land\ r\in(0,1]
\end{align*}
\end{twier}

\begin{twier}{Twierdzenie 5.17}
Niech $\seq{a}$ będzie ciagiem liczb rzeczywistych i $c\in\R_{+}\setminus\set{1}$ oraz $g\in\R$, wówczas
\begin{enumerate}[label=(\arabic*)]
    \item $\lin{a_n}=+\infty\implies\lin{a_n\log_{c}{\bigl(1+\frac{1}{a_n}\bigl)}}=\frac{1}{\ln{c}},$
    \item $\lin{a_n}=0\implies\lin{\frac{\log_c{(1+a_n)}}{a_n}}=\frac{1}{\ln{c}}$,
    \item $\lin{a_n}=g\implies\lin{log_{c}{a_n}}=\log_{c}{g}$,
    \item $\lin{a_n}=g\implies\lin{\frac{\log_{c}{a_n}-\log_{c}{g}}{a_n-g}}=\frac{1}{g\ln{c}}$.
\end{enumerate}
\end{twier}

\begin{twier}{Szacowanie funkcji eksponencjalnej i logarytmicznej 5.18}
    \begin{enumerate}[label=\Roman*.]
        \item $(\forall x\in\R)\ 1+x\le e^x\ \land\ (\forall x<1)\ e^x\le \frac{1}{1-x}$
        \item $(\forall x>-1)\ \frac{1}{1+x}\le \ln{(1+x)}\le x$
    \end{enumerate}
\end{twier}

Uwaga! Pierwsza z nierówności I. została udowodniona tylko dla $x\ge -1$, jednak można
dokonać rozszerzenia jej stosowalności. Szacowania II. wcale nie pojawiły się na wykładzie, aczkolwiek uznałem je
za przydatne. Dowody powyższych własności można znaleźć 
\href{https://drive.google.com/file/d/1cHNY7oU4XpRmYcPfO8lCgPCpdmvKQY5q/view?usp=sharing}{{\color{blue}\underline{tutaj}}}
na siódmej i dziesiątej stronie.

\begin{twier}{Twierdzenie 5.19}
Niech $\seq{a}$ będzie ciągiem liczb rzeczywistych i $c>0$, wówczas
\begin{enumerate}[label=(\arabic*)]
    \item $\lin{a_n}=0\ \land\ \forall_{n\in\N}\ a_n\neq0\implies\lin{\frac{e^{a_n}-1}{a_n}}=1$,
    \item $\lin{a_n}=0\ \land\ \forall_{n\in\N}\ a_n\neq0\implies\lin{\frac{c^{a_n}-1}{a_n}}=\ln{c}$,
    \item $\lin{a_n}=0\ \land\ \forall_{n\in\N}\ a_n\neq0\ \land\ p\in\R\implies\lin{\frac{(1+a_n)^p-1}{a_n}}=p$,
    \item $\lin{a_n}=+\infty\ \land\ p>0\implies \lin{\frac{ln{a_n}}{a_n^p}}$.
\end{enumerate}
\end{twier}

Na wykładzie własności (1) - (3) zostały udowodnione tylko dla ciągów o wyrazach dodatnich, jednakże są one stosowalne
również w przypadku ciągów o wyrazach ujemnych.

\begin{twier}{Twierdzenie Stolza 5.20}
Załóżmy, że ciąg $\seq{b}\subset\R$ jest ściśle monotoniczny oraz $(\forall n\in\N)\ b_n\neq0$. Jeśli $\seq{a}\subset\R$
i istnieje granica
\begin{equation*}
    \lin{\frac{a_{n+1}-a_{n}}{b_{n+1}-b_{n}}}=g,
\end{equation*}
a ponadto zachodzi jeden z następujących warunków:
\begin{enumerate}[label=(\roman*)]
    \item $\lin{a_n}=\lin{b_n}=0$,
    \item $\lin{b_n}=+\infty$,
\end{enumerate}
to wówczas ciąg $\frac{a_n}{b_n}$ jest zbieżny, a ponadto $\lin{\frac{a_n}{b_n}}=g$.
\end{twier}

Granica $g$ nie musi być skończona. Na wykładzie pojawił się tylko warunek (ii), dowód
zbieżności w przypadku (i) można odszukać w
\href{https://www.mimuw.edu.pl/~pawelst/analiza/Analiza_Matematyczna_1/Notatki_itp./Archiwum_files/skryptAM1-2010-11-ver01.010c.pdf}{{\color{blue}\underline{skrypcie Strzeleckiego}}}
(na str. 33) lub na
\href{https://en.wikipedia.org/wiki/Stolz%E2%80%93Ces%C3%A0ro_theorem}{{\color{blue}\underline{angielskiej wiki}}}.

\begin{twier}{Twierdzenie 5.21}
\begin{flalign*}
    c_1,\dots,c_n\in\R\ \land\ d_1,\dots,d_n>0\implies\frac{c_1+\cdots+c_n}{d_1+\cdots+d_n}&\le\max_{k\in\set{1,\dots,n}}{\frac{c_k}{d_k}}\\
    &\ge\min_{k\in\set{1,\dots,n}}{\frac{c_k}{d_k}}
\end{flalign*}
\end{twier}
\newpage


%%%%%%%%%%%%%%%%%%%%%%%%%%%%%%%%%%%%%%%%%%%%%%%%%%%%%%%%%%%%%%%%%%%%%%%%%%%%%%%%%%%%%%%%%%%%%%%%%%%%%%%%%%%%%%%%
                                \section*{Sekcja 6} \smallskip
                                {\Huge\bfseries Przestrzenie metryczne} \bigskip \medskip
%%%%%%%%%%%%%%%%%%%%%%%%%%%%%%%%%%%%%%%%%%%%%%%%%%%%%%%%%%%%%%%%%%%%%%%%%%%%%%%%%%%%%%%%%%%%%%%%%%%%%%%%%%%%%%%%

W sekcji tej $X$ będzie oznaczać dowolny niepusty zbiór, który wraz z wyróżnioną metryką $d$, tworzy przestrzeń metryczną,
chyba że w danym ustępie zaznaczę, że jest inaczej. Również pisząc o kuli, będę miał na myśli kulę otwartą.
Wszelkie niezdefiniowane $n$ będzie z domysłu liczbą naturalną.

\begin{defr}{Definicja 6.1: Metryka i przestrzeń metryczna}
    \textbf{Metryką} na zbiorze $X$ nazywa się funkcję $d:X\times X\ar\R$, spełniającą następujące warunki:
    \begin{enumerate}[label=(\arabic*)]
        \item $\forall_{x,y\in X}\hquad d(x,y)\ge 0$, 
        \item $\forall_{x,y\in X}\hquad d(x,y)=d(y,x)$,
        \item $\forall_{x,y\in X}\hquad d(x,y)=0\ifff x=y$,
        \item $\forall_{x,y,z\in X}\hquad d(x,y)+d(y,z)\ge d(x,z)$.
    \end{enumerate}
    Parę $(X,d)$, czyli zbiór $X$ z wyróżnioną metryką $d$ nazywamy \textbf{przestrzenią metryczną}.
\end{defr}

Warunek (1) jest tak na prawdę zbędny, bowiem wynika on z trzech pozostałych:
$$0\stackrel{(3)}{=}d(x,x)\stackrel{(4)}{\le} d(x,y)+d(y,x)\stackrel{(2)}{=}2d(x,y)\implies 0\le d(x,y).$$

\begin{defr}{Definicja 6.2: Kula otwarta}
    \textbf{Kulą otwartą} w przestrzeni metrycznej $(X,d)$ o środku w punkcie $a\in X$ i promieniu $r>0$
    nazywamy zbiór
    \begin{equation*}
        B(a,r)=\set{x\in X:d(a,x)<r}.
    \end{equation*}
\end{defr}

\begin{defr}{Definicja 6.3: Otoczenie}
    \textbf{Otoczeniem} punktu $a$ w przestrzeni metrycznej $(X,d)$ nazywamy zbiór $Y\subset X$, jeżeli
    \begin{equation*}
        \exists_{r>0}\hquad B(a,r)\subset Y.
    \end{equation*}
\end{defr}

\begin{defr}{Definicja 6.4: Zbieżność w przestrzeni metrycznej}
    Mówimy, że ciąg $\seq{x}\subset X$ jest zbieżny do $g$ w przestrzeni metrycznej $(X,d)$, jeśli
    \begin{enumerate}[label=(\arabic*)]
        \item $\lin{d(x_n,g)}=0$,
        \item $\forall_{\varepsilon>0}\ \exists_{M\in\N}\ \forall_{n\ge M}\hquad d(x_n,g)<\varepsilon$,
        \item dla dowolnego otoczenia $U$ punktu $g$ tylko skończona liczba elementów ciągu $\seq{x}$
        leży poza $U$ (prawie wszystkie $x_n\in U$).
    \end{enumerate}
\end{defr}

Powyższe warunki są sobie równoważne. Jeżeli ciąg spełnia jeden z nich, to spełnia wszystkie na raz.\smallskip

\textit{Uwaga! W ogólnym przypadku (dla ogólnej przesterzeni metrycznej $(X,d)$) nie ma czegoś
takiego jak rozbieżność do $\pm\infty$}.

\begin{defr}{Definicja 6.5: Ciąg ograniczony i nieograniczony}
    Ciąg $\seq{x}\subset X$ nazywamy \textbf{ograniczonym} w przestrzeni metrycznej $(X,d)$, jeżeli
    \begin{equation*}
        \exists_{a\in X}\ \exists_{r>0}\ \forall_{n\in\N}\hquad x_n\in B(a,r).
    \end{equation*}
    W przeciwnym razie ciąg $\seq{x}$ nazywamy \textbf{nieograniczonym}.
\end{defr}

\begin{defr}{Definicja 6.6: Ciąg Cauchy'ego}
    Ciąg $\seq{x}\subset X$ nazywamy \textbf{ciągiem Cauchy'ego} w przestrzeni metrycznej $(X,d)$, jeśli
    \begin{equation*}
        \forall_{\varepsilon>0}\ \exists_{M\in\N}\ \forall_{n\ge M}\hquad d(x_n,x_m)<\varepsilon.
    \end{equation*}
\end{defr}

Należy zwrócić uwagę, że istnieją przestrzenie metryczne, w których nie wszystkie ciągi Cauchy'ego są zbieżne.

\begin{defr}{Definicja 6.7: Przestrzeń zupełna}
    Przestrzeń metryczna $(X,d)$ jest \textbf{zupełna}, jeśli każdy ciąg Cauchy'ego w tej przestrzeni jest zbieżny.
\end{defr}

\begin{defr}{Definicja 6.8: Zbiór otwarty}
    Zbiór $A\subset X$ nazywamy \textbf{otwartym}, jeżeli jest otoczeniem każdego swojego punktu, czyli
    \begin{equation*}
        \forall_{x\in A}\ \exists_{r>0}\hquad B(x,r)\subset A.
    \end{equation*}
\end{defr}

\begin{defr}{Definicja 6.9: Wnętrze}
    
\end{defr}































                                                \end{document}
