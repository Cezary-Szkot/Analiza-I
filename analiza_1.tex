\documentclass{article}

\usepackage[utf8]{inputenc}
\usepackage{geometry} \geometry{margin=85pt}
\usepackage[polish]{babel}
\usepackage{polski}
\usepackage[document]{ragged2e}
\usepackage{amsmath}
\usepackage{amsthm}
\usepackage{amsfonts}
\usepackage{mathtools}
\usepackage{fancyhdr}
\usepackage{enumitem}
\usepackage[skins]{tcolorbox}
\usepackage{kpfonts}
\usepackage[T1]{fontenc}
\usepackage{xcolor}
\usepackage{hyperref}

%%%%%%%%%%
% KOLORY %
%%%%%%%%%%

\definecolor{col1}{HTML}{ffcccc}
\definecolor{col2}{HTML}{ccccff}
\definecolor{darkred}{HTML}{8B0000}

%%%%%%%%%
% RAMKI %
%%%%%%%%%

\newtcolorbox{defr}[2][]{%
  enhanced,colback=white,colframe=col1,coltitle=black,
  sharp corners,boxrule=1.5pt,
  fonttitle=\bfseries,top=13pt,
  attach boxed title to top left={yshift=-\tcboxedtitleheight/2, xshift=10pt},
  boxed title style={tile,size=small,left=5pt,right=5pt, 
  colback=col1,before upper=\strut},
  title=#2,#1}

\newtcolorbox{twier}[2][]{%
  enhanced,colback=white,colframe=col2,coltitle=black,
  sharp corners,boxrule=1.5pt,
  fonttitle=\bfseries,top=13pt,
  attach boxed title to top left={yshift=-\tcboxedtitleheight/2, xshift=10pt},
  boxed title style={tile,size=small,left=5pt,right=5pt, 
  colback=col2,before upper=\strut},
  title=#2,#1}

%%%%%%%%%%%
% KOMENDY %
%%%%%%%%%%%

\newcommand{\R}{\mathbb{R}}
\newcommand{\N}{\mathbb{N}}
\newcommand{\Q}{\mathbb{Q}}
\newcommand{\Po}{\mathcal{P}}
\newcommand{\ifff}{\Leftrightarrow}
\newcommand{\imp}{\Rightarrow}
\newcommand{\ilorazowy}[1]{#1/\!{_R}}
\DeclarePairedDelimiter\set\{\}
\ExplSyntaxOn
\NewDocumentCommand{\op}{m}
 {
  \langle
  \clist_set:Nn \l_tmpa_clist { #1 }
  \clist_use:Nn \l_tmpa_clist {,\mspace{3mu plus 1mu minus 1mu}\allowbreak}
  \rangle
}
\ExplSyntaxOff


                                \begin{document}


%%%%%%%%%%%%%%%%%%%%%%%%%%%%%%%%%%%%%%%%%%%%%%%%%%%%%%%%%%%%%%%%%%%%%%%%%%%%%%%%%%%%%%%%%%%%%%%%%%%%%%%%%%%%%%%%
                                \section*{Sekcja 1} \smallskip
                                {\Huge\bfseries Relacje} \bigskip \medskip
%%%%%%%%%%%%%%%%%%%%%%%%%%%%%%%%%%%%%%%%%%%%%%%%%%%%%%%%%%%%%%%%%%%%%%%%%%%%%%%%%%%%%%%%%%%%%%%%%%%%%%%%%%%%%%%%

\begin{defr}{Definicja 1.1: Relacja}
    Dane są dwa zbiory $A$ i $B$. \textbf{Relacją (dwuargumentową)} $R$ między elementami zbioru $A$ a elementami 
    zbioru $B$ nazywamy dowolny podzbiór iloczynu kartezjańskiego $A\times B$ ($R\subset A\times B$).  
\end{defr}

Mówimy, że elementy $a\in A$ oraz $b\in B$ są ze sobą w relacji $R$ (ozn. $a\sim b$ lub $aRb$), 
jeśli $\op{a,b}\in R$. \medskip

Niech $R$ będzie relacją na niepustym zbiorze $A$. Mówimy, że:
\begin{enumerate}[label=(\arabic*), nosep]
    \item R jest \textbf{zwrotna} $\ifff (\forall a\in A)\ aRa$.
    \item R jest \textbf{przeciwzwrotna} $\ifff(\forall a\in A)\ \lnot aRa$.
    \item R jest \textbf{przechodnia} $\ifff(\forall a,b,c\in A)(aRb \land bRc \imp aRc)$.
    \item R jest \textbf{symetryczna} $\ifff(\forall a,b\in A)(aRb\imp bRa)$.
    \item R jest \textbf{słabo antysymetryczna} $\ifff(\forall a,b\in A)(aRb\land bRa\imp a=b)$.
    \item R jest \textbf{silnie antysymetryczna} $\ifff(\forall a,b\in A)\ \lnot(aRb\land bRa)\ifff(\forall a,b\in A)
    (aRb\imp\lnot bRa)$.
    \item R jest \textbf{spójna} $\ifff(\forall a,b\in A)(aRb\lor bRa)$.
\end{enumerate} \smallskip

Relację silnie antysymetryczną nazywamy również relacją asymetryczną bądź to
przeciwsymetryczną.

\begin{defr}{Definicja 1.2: Relacja równoważności}
    Niech $R\subset A\times A$. Gdy relacja $R$ jest \textbf{zwrotna}, \textbf{symetryczna} i \textbf{przechodnia}, to mówimy, że
    jest \textbf{relacją równoważności}.
\end{defr}

\begin{defr}{Definicja 1.3: Klasa równoważności}
    Niech $R$ będzie relacją równoważności na zbiorze $A$.\textbf{ Klasą równoważności (abstrakcji)} elementu $a\in A$ względem
    relacji $R$ nazywamy zbiór
    \begin{equation*}
        [a]_R=\set{x\in A:xRa}.
    \end{equation*}
\end{defr}

\begin{defr}{Definicja 1.4: Zbiór ilorazowy}
    Zbiór wszystkich klas abstrakcji (względem) relacji równoważności $R$, czyli zbiór
    \begin{equation*}
        \ilorazowy{A}=\set{[a]_R:a\in A},
    \end{equation*}
    nazywamy \textbf{zbiorem ilorazowym} relacji $R$.
\end{defr}

\begin{twier}{Twierdzenia o klasach równoważności 1.1}
    Niech $R$ będzie relacją równoważności na zbiorze $A$. Wówczas mamy:
    \begin{enumerate}[label=(\roman*)]
        \item $(\forall a\in A)\ a\in [a]_R$ $\leftarrow$ ze zwrotności $R$
        \item $(\forall a,b\in A)(a\in [b]_R\ifff b\in [a]_R)$ $\leftarrow$ z symetryczności $R$
        \item $(\forall a,b\in A)(a\in [b]_R\ifff [a]_R=[b]_R)$
    \end{enumerate}
\end{twier}

\begin{defr}{Definicja 1.5: Relacje porządku częściowego}
    Relację $\preceq$ na zbiorze $A$ nazywamy \textbf{porządkiem częściowym 
    \textcolor{darkred}{słabym (nieostrym)}} na zbiorze $A$, jeśli jest \textbf{zwrotna},
    \textbf{przechodnia} i \textbf{słabo antysymetryczna}. \\ \smallskip
    Relację $\prec$ na zbiorze $A$ nazywamy \textbf{porządkiem częściowym \textcolor{darkred}{ostrym}} na
    zbiorze $A$, jeśli jest \textbf{przeciwzwrotna} i \textbf{przechodnia}.
\end{defr}

Na wykładzie stwierdzono, iż ostry porządek częściowy jest również asymetryczny. Jednak fakt ten wynika
już z przeciwzwrotności i przechodniości porządku, co można prosto wykazać.

\begin{proof}
    Załóżmy, że relacja $\prec$ na zbiorze $A$ jest przeciwzwrotna i przechodnia. Weżmy $a, b\in A$,
    wówczas z przechodniości
    \begin{equation*}
        a \prec b \land b \prec a \imp a \prec a \ifff \lnot(a \prec b \land b \prec a) \lor a \prec a.
    \end{equation*}
    Jednak z przeciwzwrotności $\prec$ wiemy, iż zdanie $a\prec a$ jest fałszywe dla dowolnego $a$ ze
    zbioru $A$, dlatego też
    \begin{equation*}
        \lnot(a \prec b \land b \prec a) \lor a \prec a \imp \lnot(a \prec b \land b \prec a)\ifff
        \lnot (a \prec b) \lor \lnot (b \prec a) \ifff (a\prec b\imp \lnot(b\prec a)).
    \end{equation*}
\end{proof}

\begin{defr}{Definicja 1.6: Liniowy porządek}
    Porządek częściowy $\preceq$ (lub $\prec$) na zbiorze $A$ nazywamy \textbf{porządkiem liniowym (pełnym)}
    na zbiorze $A$, jeżeli jest \textbf{spójny}.
\end{defr}

Spójność dla porządku ostrego formułujemy następująco: $(\forall a,b\in A)(a\ne b\imp a\prec b \lor b\prec a)$.

\begin{defr}{Definicja 1.7: Elementy wyróżnione}
    Dany jest zbiór $A$ z porządkiem częściowym $\preceq$. Niech $B\subset A$ i $c\in A$. Mówimy, że:
    \begin{enumerate}[label=\Roman*.]
        \item $c$ jest \textbf{ograniczeniem górnym} zbioru $B$, jeśli $(\forall b\in B)\ b \preceq c.$
        \item $c$ jest \textbf{ograniczeniem dolnym} zbioru $B$, jeśli $(\forall b\in B)\ c \preceq b.$
        \item $c$  jest \textbf{kresem górnym} (ozn. $\sup{A}$) zbioru B, jeśli:
            \begin{enumerate}
            \item jest ograniczeniem górnym.
            \item dla dowolnie innego ograniczenia górnego $c'$ zbioru $B$ zachodzi $c\preceq c'$.
            \end{enumerate}
        \item $c$  jest \textbf{kresem dolnym} (ozn. $\inf{A}$) zbioru B, jeśli:
            \begin{enumerate}
            \item jest ograniczeniem dolnym.
            \item dla dowolnie innego ograniczenia dolnego $c'$ zbioru $B$ zachodzi $c'\preceq c$.
            \end{enumerate}
        \item $c$ jest elementem \textbf{maksymalnym} zbioru $B$, jeśli $\lnot(\exists b\in B)\
        c\prec b$.
        \item $c$ jest elementem \textbf{największym} zbioru $B$, jeśli $(\forall b\in B)\ b 
        \preceq c$.
        \item $c$ jest elementem \textbf{minimalnym} zbioru $B$, jeśli $\lnot(\exists b\in B)\ 
        b\prec c$.
        \item $c$ jest elementem \textbf{najmniejszym} zbioru $B$, jeśli $(\forall b\in B)\ 
        c\preceq b$.
    \end{enumerate}
\end{defr}

Powyższe pojęcia\footnote{Na wykładzie pojawiły się wszystkie wymienione terminy, z
wyjątkiem elementu najmniejszego i największego. Zapewne dlatego, że dla porządku liniowego, który
został przyjęty, nie ma rozróżnienia między elementem największym a maksymalnym.}
na wykładzie zostały zdefiniowane tylko dla liniowo uporządkowanego zbioru $A$,
ale można je bez problemu uogólnić na zbiór z porządkiem częściowym, co też zrobiłem. Warto dodać, iż dla porządków
liniowych element największy i maksymalny znaczą to samo. Analogicznie jest z elementem najmniejszym
i minimalnym. Sprawy mają się inaczej w przypadku porządków częściowych. Oczywiście, element
największy jest również i maksymalny. Jednak implikacja w drugą stronę już nie zawsze zachodzi.
Obrazem tego stanu rzeczy są podane \href{https://calcworkshop.com/wp-content/uploads/hasse-diagram-find-the-maximal-minimal-least-and-greatest.png}{diagramy Hassego}.

\begin{twier}{Twierdzenie 1.2}
    Dane są dwie relacje $\preceq$ i $\prec$ w zbiorze $A$. Jeśli spełniają one następujące warunki:
    \begin{enumerate}[label=(\alph*)]
        \item $(\forall a,b\in A)(a\preceq b \ifff a\prec b \lor a=b)$
        \item $(\forall a,b\in A)(a\prec b \ifff a\preceq b \land a\ne b)$,
    \end{enumerate}
    wówczas $\preceq$ jest porządkiem słabym, wtedy i tylko wtedy gdy $\prec$ jest porządkiem ostrym. 
\end{twier}

\begin{defr}{Definicja 1.8: Relacja odwrotna}
    Niech $R\subset A\times B$. \textbf{Relacją odwrotną} $R^{-1}$ do relacji $R$ nazywamy zbiór
    \begin{equation*}
        R^{-1}:=\set{\op{a,b}\in A\times B:\op{a,b}\in R}.
    \end{equation*}
    Innymi słowy $(\forall a\in A)(\forall b\in B)(bR^{-1}a\ifff aRb)$.
\end{defr}

\newpage


%%%%%%%%%%%%%%%%%%%%%%%%%%%%%%%%%%%%%%%%%%%%%%%%%%%%%%%%%%%%%%%%%%%%%%%%%%%%%%%%%%%%%%%%%%%%%%%%%%%%%%%%%%%%%%%%
                                \section*{Sekcja 2} \smallskip
                                {\Huge\bfseries Funkcje} \bigskip \medskip
%%%%%%%%%%%%%%%%%%%%%%%%%%%%%%%%%%%%%%%%%%%%%%%%%%%%%%%%%%%%%%%%%%%%%%%%%%%%%%%%%%%%%%%%%%%%%%%%%%%%%%%%%%%%%%%%

\begin{defr}{Definicja 2.1: Funkcja}
    Relację $f$ między elementami zbioru $A$ i elementami zbioru $B$ nazywamy \textbf{funkcją}, jeżeli
    \begin{equation*}
        (\forall x\in A)(\exists! y\in B)\ \op{x,y}\in f.
    \end{equation*}
    Powyższe zdanie można zapisać równoważnie jako
    \begin{equation*}
        (\forall x\in A)(\exists y\in B)\ \op{x,y}\in f\ \land\ ((\forall x\in A)(\forall y_1,y_2\in B)
        (\op{x,y_1}\in f\land \op{x,y_2}\in f\imp y_1=y_2)).
    \end{equation*}
\end{defr}

\begin{defr}{Definicja 2.2: Injekcja}
    Relację funkcyjną $f\subset A\times B$ nazywamy \textbf{injekcją} (różnowartościową), jeżeli
    \begin{gather*}
        (\forall x_1,x_2\in A)(f(x_1)=f(x_2)\imp x_1=x_2)\text{, czyli równoważnie}\\
        (\forall x_1,x_2\in A)(\forall y\in B)(\op{x_1,y}\in f \land \op{x_2,y}\in f\imp x_1=x_2).
    \end{gather*}
\end{defr}

\begin{defr}{Definicja 2.3: Surjekcja}
    Mówimy, że relacja funkcyjna $f\subset A\times B$ jest ze zbioru $A$ \textbf{na} zbiór $B$, jeśli
    \begin{equation*}
        (\forall y\in B)(\exists x\in A)\ y=f(x)\text{, czyli }(\forall y\in B)(\exists x\in A)\ 
        \op{x,y}\in f.
    \end{equation*}
    Funkcję taką nazywamy też \textbf{surjekcją}.
\end{defr}

\begin{defr}{Definicja 2.4: Bijekcja}
    Relację funkcyjną, która jest zarówno injekcją jak i surjekcją nazywamy \textbf{bijekcją}.
\end{defr}

\begin{defr}{Definicja 2.5: Funkcja odwrotna}
    Jeśli $f:A\rightarrow B$ jest bijekcją, to \textbf{funkcją odwrotną} do $f$
    jest funkcja $f^{-1}:B\rightarrow A$, taka że
    \begin{equation*}
        (\forall x\in A)(\forall y\in B)(\op{y,x}\in f^{-1}\ifff \op{x,y}\in f)
    \end{equation*}
\end{defr}

\begin{twier}{Twierdzenie 2.1}
    \begin{enumerate}[label=(\arabic*), nosep]
        \item Jeżeli funkcja jest bijekcją, to posiada funkcję odwrotną, 
        która również jest bijekcją
        \item Jeżeli funkcja jest odwracalna, to oznacza, że jest bijekcją.
    \end{enumerate}
\end{twier}

\begin{defr}{Definicja 2.6: Złożenie Funkcji}
    Niech $f:A\to B$, $g:B\to C$ i $x\in A$. \textbf{Złożeniem funkcji} $f$ z funkcją $g$ nazywamy
    funkcję $g\circ f:A\to C$, określoną wzorem $(g\circ f)(x)=g(f(x))$.
\end{defr}

Wyrażenie $(g\circ f)(x)=g(f(x))$ można zapisać alternatywnie jako:
\begin{equation*}
    (\forall x\in A)(\forall z\in C)\ \op{x,z}\in g\circ f \ifff (\exists y\in B)
    (\op{x,y}\in f\land \op{y,z}\in g).
\end{equation*}

\begin{twier}{Twierdzenie 2.2}
    Dla dowolnych funkcji $f$, $g$, $h$ zachodzi równość
     $(f\circ g)\circ h= f\circ (g\circ h)$.
\end{twier}

\begin{defr}{Definicja 2.7: Funkcja identycznościowa}
    Dla dowolnego niepustego zbioru $A$ możemy określić \textbf{funkcję identycznościową}
    na zbiorze $A$ (identyczność na zbiorze $A$) następująco:
    \begin{equation*}
        id_A:A\to A,\quad (\forall x\in A)\ id_A(x)=x.
    \end{equation*}
\end{defr}

\begin{twier}{Twierdzenie 2.3}
    Jeśli $f:A\to B$ i $f^{-1}:B\to A$, to $f^{-1}\circ f:A\to A$ jest identycznością na
    zbiorze $A$.
\end{twier}

\newpage


%%%%%%%%%%%%%%%%%%%%%%%%%%%%%%%%%%%%%%%%%%%%%%%%%%%%%%%%%%%%%%%%%%%%%%%%%%%%%%%%%%%%%%%%%%%%%%%%%%%%%%%%%%%%%%%%
                                \section*{Sekcja 3}\smallskip
                                {\Huge\bfseries Równoliczność} \bigskip \medskip
%%%%%%%%%%%%%%%%%%%%%%%%%%%%%%%%%%%%%%%%%%%%%%%%%%%%%%%%%%%%%%%%%%%%%%%%%%%%%%%%%%%%%%%%%%%%%%%%%%%%%%%%%%%%%%%%

\begin{defr}{Definicja 3.1: Równoliczność}
    Mówimy, że zbiory $A$ i $B$ są \textbf{równoliczne} (ozn. $|A|=|B|$, $A\sim B$),
    gdy istnieje bijekcja $f:A\to B$.
\end{defr}

Równoliczność \textit{ma własności relacji równoważności} 
(jest zwrotna, symetryczna i przechodnia)
i faktycznie nią jest, gdy ograniczymy relację równoliczności do zbioru $\Po(U)$
\footnote{$\Po(U)$ to zbiór potęgowy pewnego zbioru $U$, czyli zbiór wszystkich
podzbiorów $U$. Równoliczność ograniczamy do jakiegoś
zbioru potęgowego, bo jej dziedzina i obraz nie są normalnie zbiorami,
 więc nie byłaby ona relacją równoważności w ścisłym sensie.}. 
 Jeśli $A, B\in \Po(U)$ i $R$ będzie symbolizować relację równoliczności, 
 to możemy przyjąć, iż $|A|=|B|$ oznacza, że
$[A]_R=[B]_R$.

\begin{defr}{Definicja 3.2: Zbiór skończony}
    O zbiorze $A$ mówimy, że jest \textbf{skończony}, jeżeli jest pusty lub równoliczny
    jakiemuś zbiorowi postaci $\set{1,\ldots,n}_{n\in\N}$. Piszemy wówczas, że $|A|=n$
\end{defr}

\begin{defr}{Definicja 3.3: Zbiór przeliczalny}
    Mówimy, że zbiór $A$ jest \textbf{przeliczalny}, jeżeli jest równoliczny zbiorowi
    $\N$. Piszemy wówcza, że $|A|=\aleph_0$.
\end{defr}

\begin{defr}{Definicja 3.4: Zbiór nieprzeliczalny}
    Mówimy, że zbiór $A$ jest \textbf{nieprzeliczalny}, jeżeli nie jest przeliczalny,
    ani skończony.
\end{defr}

Zbiór liczb rzeczywistych jest zbiorem nieprzeliczalnym. |$\R$| oznaczamy 
jako $\mathfrak{c}$ lub $2^{\aleph_0}$ i nazywamy \textit{continuum}. Continuum jest
większe od mocy $\N$.





















                                                \end{document}
