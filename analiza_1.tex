\documentclass{article}

\usepackage[utf8]{inputenc}
\usepackage{geometry} \geometry{margin=85pt,tmargin=55pt}
\usepackage[polish]{babel}
\usepackage[OT4]{fontenc}
\usepackage[document]{ragged2e}
\usepackage{amsmath}
\usepackage{amsthm}
\usepackage{amsfonts}
\usepackage{mathtools}
\usepackage{fancyhdr}
\usepackage{enumitem}
\usepackage[skins]{tcolorbox}
\usepackage{kpfonts}
\usepackage[T1]{fontenc}
\usepackage{xcolor}
\usepackage{hyperref}
\usepackage{graphicx}
\usepackage{braket}
\usepackage{physics}

\preto\equation{\setcounter{equation}{0}}
\makeatletter
\pretocmd\start@gather{\setcounter{equation}{0}}{}{}
\pretocmd\start@align{\setcounter{equation}{0}}{}{}
\pretocmd\start@multline{\setcounter{equation}{0}}{}{}
\makeatother


%%%%%%%%%%%%%%%
% TWIERDZENIA $
%%%%%%%%%%%%%%%

\newtheorem*{dedekind}{{\color{darkred} Aksjomat ciągłości (Dedekinda)}}

%%%%%%%%%%
% KOLORY %
%%%%%%%%%%

\definecolor{col1}{HTML}{ffcccc}
\definecolor{col2}{HTML}{ccccff}
\definecolor{darkred}{HTML}{8B0000}
\definecolor{goldenrod}{HTML}{FFDF42}
\definecolor{lightgoldenrod}{HTML}{fff0a5}

%%%%%%%%%
% RAMKI %
%%%%%%%%%

\newtcolorbox{defr}[2][]{%
  enhanced,colback=white,colframe=col1,coltitle=black,
  sharp corners,boxrule=1.5pt,
  fonttitle=\bfseries,top=13pt,
  attach boxed title to top left={yshift=-\tcboxedtitleheight/2, xshift=10pt},
  boxed title style={tile,size=small,left=5pt,right=5pt, 
  colback=col1,before upper=\strut},
  title=#2,#1}

\newtcolorbox{twier}[2][]{%
  enhanced,colback=white,colframe=col2,coltitle=black,
  sharp corners,boxrule=1.5pt,
  fonttitle=\bfseries,top=13pt,
  attach boxed title to top left={yshift=-\tcboxedtitleheight/2, xshift=10pt},
  boxed title style={tile,size=small,left=5pt,right=5pt, 
  colback=col2,before upper=\strut},
  title=#2,#1}

\newtcolorbox{wn}[2][]{%
  enhanced,colback=white,colframe=white,coltitle=black,
  sharp corners,boxrule=1.5pt,
  fonttitle=\bfseries,top=13pt,
  attach boxed title to top left={yshift=-\tcboxedtitleheight/2, xshift=0pt},
  boxed title style={tile,size=small,left=5pt,right=5pt, 
  colback=lightgoldenrod,before upper=\strut},
  title=#2,#1}

%%%%%%%%%%%
% KOMENDY %
%%%%%%%%%%%

\newcommand{\hquad}{\mkern9mu}
\newcommand{\R}{\mathbb{R}}
\newcommand{\N}{\mathbb{N}}
\newcommand{\Q}{\mathbb{Q}}
\newcommand{\Po}{\mathcal{P}}
\newcommand{\ifff}{\Leftrightarrow}
\newcommand{\imp}{\Rightarrow}
\newcommand{\ilorazowy}[1]{#1/\!{_R}}
\newcommand{\lin}[1]{\lim\limits_{n\to\infty}{#1}}
\newcommand{\arn}{\xrightarrow{n\to\infty}}
\newcommand{\arr}[1]{\xrightarrow{#1}}
\newcommand{\ar}{\rightarrow}
\newcommand*\closure[1]{\overline{#1}}
\newcommand*\Interior[1]{\mathring{#1}}

\DeclareMathOperator{\interior}{int}
\DeclareMathOperator{\Closure}{cl}
\DeclareMathOperator{\arccosh}{arccosh}
\DeclareMathOperator{\arcsinh}{arcsinh}
\newcommand{\sgn}{\mathop{\mathrm{sgn}}}

\ExplSyntaxOn
\RenewDocumentCommand{\op}{m}
 {
  \langle
  \clist_set:Nn \l_tmpa_clist { #1 }
  \clist_use:Nn \l_tmpa_clist {,\mspace{3mu plus 1mu minus 1mu}\allowbreak}
  \rangle
}
\ExplSyntaxOff
\newcommand{\seq}[1]{\set{#1_n}_{n\in\N}}
\newcommand{\ri}[1]{\set{#1_i}_{i\in I}}

\renewcommand{\arraystretch}{1.8}


                                \begin{document}


%%%%%%%%%%%%%%%%%%%%%%%%%%%%%%%%%%%%%%%%%%%%%%%%%%%%%%%%%%%%%%%%%%%%%%%%%%%%%%%%%%%%%%%%%%%%%%%%%%%%%%%%%%%%%%%%
                                \section*{Sekcja 1} \smallskip
                                {\Huge\bfseries Relacje} \bigskip \medskip
%%%%%%%%%%%%%%%%%%%%%%%%%%%%%%%%%%%%%%%%%%%%%%%%%%%%%%%%%%%%%%%%%%%%%%%%%%%%%%%%%%%%%%%%%%%%%%%%%%%%%%%%%%%%%%%%

\begin{defr}{Definicja 1.1: Relacja}
    Dane są dwa zbiory $A$ i $B$. \textbf{Relacją (dwuargumentową)} $R$ między elementami zbioru $A$ a elementami 
    zbioru $B$ nazywamy dowolny podzbiór iloczynu kartezjańskiego $A\times B$ ($R\subset A\times B$).  
\end{defr}

Mówimy, że elementy $a\in A$ oraz $b\in B$ są ze sobą w relacji $R$ (ozn. $a\sim b$ lub $aRb$), 
jeśli $\op{a,b}\in R$. \medskip

Niech $R$ będzie relacją na niepustym zbiorze $A$. Mówimy, że:
\begin{enumerate}[label=(\arabic*), nosep]
    \item R jest \textbf{zwrotna} $\ifff (\forall a\in A)\ aRa$.
    \item R jest \textbf{przeciwzwrotna} $\ifff(\forall a\in A)\ \lnot aRa$.
    \item R jest \textbf{przechodnia} $\ifff(\forall a,b,c\in A)(aRb \land bRc \imp aRc)$.
    \item R jest \textbf{symetryczna} $\ifff(\forall a,b\in A)(aRb\imp bRa)$.
    \item R jest \textbf{słabo antysymetryczna} $\ifff(\forall a,b\in A)(aRb\land bRa\imp a=b)$.
    \item R jest \textbf{silnie antysymetryczna} $\ifff(\forall a,b\in A)\ \lnot(aRb\land bRa)\ifff(\forall a,b\in A)
    (aRb\imp\lnot bRa)$.
    \item R jest \textbf{spójna} $\ifff(\forall a,b\in A)(aRb\lor bRa)$.
\end{enumerate} \smallskip

Relację silnie antysymetryczną nazywamy również relacją asymetryczną bądź to
przeciwsymetryczną.

\begin{defr}{Definicja 1.2: Relacja równoważności}
    Niech $R\subset A\times A$. Gdy relacja $R$ jest \textbf{zwrotna}, \textbf{symetryczna} i \textbf{przechodnia}, to mówimy, że
    jest \textbf{relacją równoważności}.
\end{defr}

\begin{defr}{Definicja 1.3: Klasa równoważności}
    Niech $R$ będzie relacją równoważności na zbiorze $A$.\textbf{ Klasą równoważności (abstrakcji)} elementu $a\in A$ względem
    relacji $R$ nazywamy zbiór
    \begin{equation*}
        [a]_R=\set{x\in A:xRa}.
    \end{equation*}
\end{defr}

\begin{defr}{Definicja 1.4: Zbiór ilorazowy}
    Zbiór wszystkich klas abstrakcji (względem) relacji równoważności $R$, czyli zbiór
    \begin{equation*}
        \ilorazowy{A}=\set{[a]_R:a\in A},
    \end{equation*}
    nazywamy \textbf{zbiorem ilorazowym} relacji $R$.
\end{defr}

\begin{twier}{Twierdzenia o klasach równoważności 1.1}
    Niech $R$ będzie relacją równoważności na zbiorze $A$. Wówczas mamy:
    \begin{enumerate}[label=(\roman*)]
        \item $(\forall a\in A)\ a\in [a]_R$ $\leftarrow$ ze zwrotności $R$
        \item $(\forall a,b\in A)(a\in [b]_R\ifff b\in [a]_R)$ $\leftarrow$ z symetryczności $R$
        \item $(\forall a,b\in A)(a\in [b]_R\ifff [a]_R=[b]_R)$
    \end{enumerate}
\end{twier}

\begin{defr}{Definicja 1.5: Relacje porządku częściowego}
    Relację $\preceq$ na zbiorze $A$ nazywamy \textbf{porządkiem częściowym 
    \textcolor{darkred}{słabym (nieostrym)}} na zbiorze $A$, jeśli jest \textbf{zwrotna},
    \textbf{przechodnia} i \textbf{słabo antysymetryczna}. \\ \smallskip
    Relację $\prec$ na zbiorze $A$ nazywamy \textbf{porządkiem częściowym \textcolor{darkred}{ostrym}} na
    zbiorze $A$, jeśli jest \textbf{przeciwzwrotna} i \textbf{przechodnia}.
\end{defr}

Na wykładzie stwierdzono, iż ostry porządek częściowy jest również asymetryczny. Jednak fakt ten wynika
już z przeciwzwrotności i przechodniości porządku, co można prosto wykazać.

\begin{proof}
    Załóżmy, że relacja $\prec$ na zbiorze $A$ jest przeciwzwrotna i przechodnia. Weżmy $a, b\in A$,
    wówczas z przechodniości
    \begin{equation*}
        a \prec b \land b \prec a \imp a \prec a \ifff \lnot(a \prec b \land b \prec a) \lor a \prec a.
    \end{equation*}
    Jednak z przeciwzwrotności $\prec$ wiemy, iż zdanie $a\prec a$ jest fałszywe dla dowolnego $a$ ze
    zbioru $A$, dlatego też
    \begin{equation*}
        \lnot(a \prec b \land b \prec a) \lor a \prec a \imp \lnot(a \prec b \land b \prec a)\ifff
        \lnot (a \prec b) \lor \lnot (b \prec a) \ifff (a\prec b\imp \lnot(b\prec a)).
    \end{equation*}
\end{proof}

\begin{defr}{Definicja 1.6: Liniowy porządek}
    Porządek częściowy $\preceq$ (lub $\prec$) na zbiorze $A$ nazywamy \textbf{porządkiem liniowym (pełnym)}
    na zbiorze $A$, jeżeli jest \textbf{spójny}.
\end{defr}

Spójność dla porządku ostrego formułujemy następująco: $(\forall a,b\in A)(a\ne b\imp a\prec b \lor b\prec a)$.

\begin{defr}{Definicja 1.7: Elementy wyróżnione}
    Dany jest zbiór $A$ z porządkiem częściowym $\preceq$. Niech $B\subset A$ i $c\in A$. Mówimy, że:
    \begin{enumerate}[label=\Roman*.]
        \item $c$ jest \textbf{ograniczeniem górnym} zbioru $B$, jeśli $(\forall b\in B)\ b \preceq c.$
        \item $c$ jest \textbf{ograniczeniem dolnym} zbioru $B$, jeśli $(\forall b\in B)\ c \preceq b.$
        \item $c$  jest \textbf{kresem górnym} (ozn. $\sup{A}$) zbioru B, jeśli:
            \begin{enumerate}
            \item jest ograniczeniem górnym.
            \item dla dowolnie innego ograniczenia górnego $c'$ zbioru $B$ zachodzi $c\preceq c'$.
            \end{enumerate}
        \item $c$  jest \textbf{kresem dolnym} (ozn. $\inf{A}$) zbioru B, jeśli:
            \begin{enumerate}
            \item jest ograniczeniem dolnym.
            \item dla dowolnie innego ograniczenia dolnego $c'$ zbioru $B$ zachodzi $c'\preceq c$.
            \end{enumerate}
        \item $c$ jest elementem \textbf{maksymalnym} zbioru $B$, jeśli $c\in B \land
        \lnot(\exists b\in B)\
        c\prec b$.
        \item $c$ jest elementem \textbf{największym} zbioru $B$, jeśli $c\in B \land
        (\forall b\in B)\ b 
        \preceq c$.
        \item $c$ jest elementem \textbf{minimalnym} zbioru $B$, jeśli $c\in B \land
        \lnot(\exists b\in B)\ 
        b\prec c$.
        \item $c$ jest elementem \textbf{najmniejszym} zbioru $B$, jeśli $c\in B \land
        (\forall b\in B)\ 
        c\preceq b$.
    \end{enumerate}
\end{defr}

Powyższe pojęcia\footnote{Na wykładzie pojawiły się wszystkie wymienione terminy, z
wyjątkiem elementu najmniejszego i największego. Zapewne dlatego, że dla porządku liniowego, który
został przyjęty, nie ma rozróżnienia między elementem największym a maksymalnym.}
na wykładzie zostały zdefiniowane tylko dla liniowo uporządkowanego zbioru $A$,
ale można je bez problemu uogólnić na zbiór z porządkiem częściowym, co też zrobiłem. Warto dodać, iż dla porządków
liniowych element największy i maksymalny znaczą to samo. Analogicznie jest z elementem najmniejszym
i minimalnym. Sprawy mają się inaczej w przypadku porządków częściowych. Oczywiście, element
największy jest również i maksymalny. Jednak implikacja w drugą stronę już nie zawsze zachodzi.
Obrazem tego stanu rzeczy są podane \href{https://calcworkshop.com/wp-content/uploads/hasse-diagram-find-the-maximal-minimal-least-and-greatest.png}{{\color{blue} \underline{diagramy Hassego}}}.

\begin{twier}{Twierdzenie 1.2}
    Dane są dwie relacje $\preceq$ i $\prec$ w zbiorze $A$. Jeśli spełniają one następujące warunki:
    \begin{enumerate}[label=(\alph*)]
        \item $(\forall a,b\in A)(a\preceq b \ifff a\prec b \lor a=b)$
        \item $(\forall a,b\in A)(a\prec b \ifff a\preceq b \land a\ne b)$,
    \end{enumerate}
    wówczas $\preceq$ jest porządkiem słabym wtedy i tylko wtedy, gdy $\prec$ jest porządkiem ostrym. 
\end{twier}

\begin{defr}{Definicja 1.8: Relacja odwrotna}
    Niech $R\subset A\times B$. \textbf{Relacją odwrotną} $R^{-1}$ do relacji $R$ nazywamy zbiór
    \begin{equation*}
        R^{-1}:=\set{\op{a,b}\in A\times B:\op{a,b}\in R}.
    \end{equation*}
    Innymi słowy $(\forall a\in A)(\forall b\in B)(bR^{-1}a\ifff aRb)$.
\end{defr}

\newpage


%%%%%%%%%%%%%%%%%%%%%%%%%%%%%%%%%%%%%%%%%%%%%%%%%%%%%%%%%%%%%%%%%%%%%%%%%%%%%%%%%%%%%%%%%%%%%%%%%%%%%%%%%%%%%%%%
                                \section*{Sekcja 2} \smallskip
                                {\Huge\bfseries Funkcje} \bigskip \medskip
%%%%%%%%%%%%%%%%%%%%%%%%%%%%%%%%%%%%%%%%%%%%%%%%%%%%%%%%%%%%%%%%%%%%%%%%%%%%%%%%%%%%%%%%%%%%%%%%%%%%%%%%%%%%%%%%

\begin{defr}{Definicja 2.1: Funkcja}
    Relację $f$ między elementami zbioru $A$ i elementami zbioru $B$ nazywamy \textbf{funkcją}, jeżeli
    \begin{equation*}
        (\forall x\in A)(\exists! y\in B)\ \op{x,y}\in f.
    \end{equation*}
    Powyższe zdanie można zapisać równoważnie jako
    \begin{equation*}
        (\forall x\in A)(\exists y\in B)\ \op{x,y}\in f\ \land\ ((\forall x\in A)(\forall y_1,y_2\in B)
        (\op{x,y_1}\in f\land \op{x,y_2}\in f\imp y_1=y_2)).
    \end{equation*}
    \begin{itemize}
        \item \textbf{Dziedziną} (ozn. $dom(f)$ lub $D_f$) funkcji $f$ nazywamy zbiór $A$.
        \item \textbf{Przeciwdziedziną} (ozn. $\rotatebox[origin=c]{180}{$D$}_f$)
        funkcji $f$ nazywamy zbiór $B$.
        \item \textbf{Zbiorem wartości} (ozn. $rng(f)$ lub $R_f$) funkcji $f$ nazywamy zbiór \\
        $R_f = \set{y\in B: (\exists x\in A)\ \op{x,y}\in f}\subset B$.
    \end{itemize}
\end{defr}

\begin{defr}{Definicja 2.2: Obraz i przeciwobraz}
    Niech $f:A\to B$ oraz $C\subset A$ i $D\subset B$.
    \begin{enumerate}[label=(\arabic*)]
        \item \textbf{Obrazem} zbioru $C$ względem funkcji $f$ nazywamy zbiór
        \begin{equation*}
            f[C]=\set{y\in B: (\exists x\in C)\ y=f(x)}=\set{f(x):x\in C}.
        \end{equation*}
        \item \textbf{Przeciwobrazem} zbioru $D$ względem funkcji $f$ nazywamy zbiór
        \begin{equation*}
            f^{-1}[D]=\set{x\in A:f(x)\in D}.
        \end{equation*}
    \end{enumerate}
\end{defr}

Supremum funkcji $f$ na zbiorze $C$ $\sup_{x\in C}f(x)$ jest kresem górnym obrazu zbioru $C$
względem niej. Analogicznie definiujemy $\inf_{x\in C}f(x)$, $\max_{x\in C}f(x)$ i
$\min_{x\in C}f(x)$.

\begin{defr}{Definicja 2.3: Injekcja}
    Relację funkcyjną $f\subset A\times B$ nazywamy \textbf{injekcją} (różnowartościową), jeżeli
    \begin{gather*}
        (\forall x_1,x_2\in A)(f(x_1)=f(x_2)\imp x_1=x_2)\text{, czyli równoważnie}\\
        (\forall x_1,x_2\in A)(\forall y\in B)(\op{x_1,y}\in f \land \op{x_2,y}\in f\imp x_1=x_2).
    \end{gather*}
\end{defr}

\begin{defr}{Definicja 2.4: Surjekcja}
    Mówimy, że relacja funkcyjna $f\subset A\times B$ jest ze zbioru $A$ \textbf{na} zbiór $B$, jeśli
    \begin{equation*}
        (\forall y\in B)(\exists x\in A)\ y=f(x)\text{, czyli }(\forall y\in B)(\exists x\in A)\ 
        \op{x,y}\in f.
    \end{equation*}
    Funkcję taką nazywamy też \textbf{surjekcją}.
\end{defr}

\begin{defr}{Definicja 2.5: Bijekcja}
    Relację funkcyjną, która jest zarówno injekcją jak i surjekcją nazywamy \textbf{bijekcją}.
\end{defr}

\begin{defr}{Definicja 2.6: Funkcja odwrotna}
    Jeśli $f:A\rightarrow B$ jest bijekcją, to \textbf{funkcją odwrotną} do $f$
    jest funkcja $f^{-1}:B\rightarrow A$, taka że
    \begin{equation*}
        (\forall x\in A)(\forall y\in B)(\op{y,x}\in f^{-1}\ifff \op{x,y}\in f)
    \end{equation*}
\end{defr}

\begin{twier}{Twierdzenie 2.1}
    \begin{enumerate}[label=(\arabic*), nosep]
        \item Jeżeli funkcja jest bijekcją, to posiada funkcję odwrotną, 
        która również jest bijekcją
        \item Jeżeli funkcja jest odwracalna, to oznacza, że jest bijekcją.
    \end{enumerate}
\end{twier}

\begin{defr}{Definicja 2.7: Złożenie Funkcji}
    \hfill \texttt{WERSJA I}\\ \smallskip
    Niech $f:A\to B$, $g:B\to C$ (wystarczy nawet założyć, że
    $g:B_1\to C$, jesli $B\subset B_1$)
    i $x\in A$. \textbf{Złożeniem funkcji} $f$ z funkcją $g$ nazywamy
    funkcję $g\circ f:A\to C$, określoną wzorem $(g\circ f)(x)=g(f(x))$.
    \\ \medskip
    \hfill \texttt{WERSJA II}\\ \smallskip
    \textbf{Złożeniem funkcji} $f$ i $g$ nazywamy funkcję $g\circ f$
    zdefiniowaną wzorem
    \begin{equation*}
        g\circ f=\set{\op{x,z}\in D_f\times R_g:\exists y[\op{x,y}\in f
        \land \op{y,z}\in g]}.
    \end{equation*}
\end{defr}

Wyrażenie $(g\circ f)(x)=g(f(x))$ można zapisać alternatywnie jako:
\begin{equation*}
    (\forall x\in A)(\forall z\in C)\ \op{x,z}\in g\circ f \ifff (\exists y\in B)
    (\op{x,y}\in f\land \op{y,z}\in g).
\end{equation*}

Zauważmy różnicę między tymi dwiema wersjami
\footnote{Tylko wersja pierwsza pojawiła się na wykładzie. Ta druga jest
tylko moim dodatkiem. Podaję ją tutaj, bo chociaż nie pojawiła się w czasie
wykładu, to posługiwaliśmy się jej wyróżniającą własnością na ćwiczeniach.}
definicji złożenia funkcji.
Pierwsza, częściej spotykana, zakłada że $R_f\subset D_g$, skąd wynika, że
$D_{g\circ f}=D_f$. Natomiast według drugiej definicji, złożenie $g\circ f$
funkcji $f$ i $g$ ma następujące własności.
\begin{enumerate}[label=(\alph*)]
    \item $D_{g\circ f}=\set{x\in D_f:f(x)\in D_g}$,
    \item $(\forall x\in D_{g\circ f})\ (g\circ f)(x)=g(f(x))$.
\end{enumerate} 

\begin{twier}{Twierdzenie 2.2}
    Dla dowolnych funkcji $f$, $g$, $h$ zachodzi równość
     $(f\circ g)\circ h= f\circ (g\circ h)$.
\end{twier}

\begin{defr}{Definicja 2.8: Funkcja identycznościowa}
    Dla dowolnego niepustego zbioru $A$ możemy określić \textbf{funkcję identycznościową}
    na zbiorze $A$ (identyczność na zbiorze $A$) następująco:
    \begin{equation*}
        id_A:A\to A,\quad (\forall x\in A)\ id_A(x)=x.
    \end{equation*}
\end{defr}

\begin{twier}{Twierdzenie 2.3}
    Jeśli $f:A\to B$ i $f^{-1}:B\to A$, to $f^{-1}\circ f:A\to A$ jest identycznością na
    zbiorze $A$.
\end{twier}

\begin{defr}{Definicja 2.9: Obcięcie i przedłużenie funkcji}
    Niech $f:A\to B$.
    \begin{enumerate}[label=(\arabic*)]
        \item Niech $C\subset X$. \textbf{Obcięciem funkcji} $f$ do zbioru
        $C$ nazywamy funkcję $f|_C:C\to B$,  $(f|_C)(x)=f(x)$.
        \item Funkcję $g:C\to B$ nazywamy \textbf{przedłużeniem funkcji}
        $f$, jeśli $A\subset C$ oraz $(\forall x\in A)\ f(x)=g(x)$.
    \end{enumerate}
\end{defr}

Zwróćmy uwagę, że daną funkcję $f$ można przedłużyć na dany właściwy nadzbiór
jej dziedziny na różne sposoby. Zauważmy też, że funkcja jest zawsze przedłużeniem
swojego obcięcia.

\newpage


%%%%%%%%%%%%%%%%%%%%%%%%%%%%%%%%%%%%%%%%%%%%%%%%%%%%%%%%%%%%%%%%%%%%%%%%%%%%%%%%%%%%%%%%%%%%%%%%%%%%%%%%%%%%%%%%
                                \section*{Sekcja 3}\smallskip
                                {\Huge\bfseries Równoliczność} \bigskip \medskip
%%%%%%%%%%%%%%%%%%%%%%%%%%%%%%%%%%%%%%%%%%%%%%%%%%%%%%%%%%%%%%%%%%%%%%%%%%%%%%%%%%%%%%%%%%%%%%%%%%%%%%%%%%%%%%%%

\begin{defr}{Definicja 3.1: Równoliczność}
    Mówimy, że zbiory $A$ i $B$ są \textbf{równoliczne} (ozn. $|A|=|B|$, $A\sim B$),
    gdy istnieje bijekcja $f:A\to B$.
\end{defr}

Równoliczność \textit{ma własności relacji równoważności} 
(jest zwrotna, symetryczna i przechodnia)
i faktycznie nią jest, gdy ograniczymy relację równoliczności do zbioru $\Po(U)$
\footnote{$\Po(U)$ to zbiór potęgowy pewnego zbioru $U$, czyli zbiór wszystkich
podzbiorów $U$. Równoliczność ograniczamy do jakiegoś
zbioru potęgowego, bo jej dziedzina i obraz nie są normalnie zbiorami,
 więc nie byłaby ona relacją równoważności w ścisłym sensie.}. 
 Jeśli $A, B\in \Po(U)$ i $R$ będzie symbolizować relację równoliczności, 
 to możemy przyjąć, iż $|A|=|B|$ oznacza, że
$[A]_R=[B]_R$.

\begin{defr}{Definicja 3.2: Zbiór skończony i nieskończony}
    O zbiorze $A$ mówimy, że jest \textbf{skończony}, jeżeli jest pusty lub równoliczny
    jakiemuś zbiorowi postaci $\set{1,\ldots,n}_{n\in\N}$. Piszemy wówczas, że $|A|=n$.
    Zbiór, który nie jest skończony nazywamy \textbf{nieskończonym}.
\end{defr}

\begin{defr}{Definicja 3.3: Zbiór przeliczalny}
    Mówimy, że zbiór $A$ jest \textbf{przeliczalny}, jeżeli jest równoliczny zbiorowi
    $\N$. Piszemy wówczas, że $|A|=\aleph_0$. 
\end{defr}

Zbiór nazywamy \textit{co najwyżej przeliczalnym}, jeśli jest on skończony lub przeliczalny.

\begin{defr}{Definicja 3.4: Zbiór nieprzeliczalny}
    Mówimy, że zbiór $A$ jest \textbf{nieprzeliczalny}, jeżeli nie jest przeliczalny,
    ani skończony.
\end{defr}

Zbiór liczb rzeczywistych jest zbiorem nieprzeliczalnym. |$\R$| oznaczamy 
jako $\mathfrak{c}$ lub $2^{\aleph_0}$ i nazywamy \textit{continuum}. Continuum jest
większe od mocy $\N$.

\newpage


%%%%%%%%%%%%%%%%%%%%%%%%%%%%%%%%%%%%%%%%%%%%%%%%%%%%%%%%%%%%%%%%%%%%%%%%%%%%%%%%%%%%%%%%%%%%%%%%%%%%%%%%%%%%%%%%
                                \section*{Sekcja 4} \smallskip
                                {\Huge\bfseries Liczby rzeczywiste} \bigskip \medskip
%%%%%%%%%%%%%%%%%%%%%%%%%%%%%%%%%%%%%%%%%%%%%%%%%%%%%%%%%%%%%%%%%%%%%%%%%%%%%%%%%%%%%%%%%%%%%%%%%%%%%%%%%%%%%%%%

Dany jest zbiór liczb rzeczywistych \(\R\) z określonymi działaniami $+$ i $\cdot$
oraz porządkiem liniowym $\le$.
Aksjomaty teorii liczb rzeczywistych podzielimy na
trzy kategorie: aksjomaty ciała przemiennego, aksjomaty porządku, oraz aksjomat
ciągłości\footnote{Aksjomaty przepisałem ze skryptu Strzeleckiego, ponieważ
były tam zapisane w trochę bardziej eleganckiej postaci.}.

\subsection*{{\color{darkred} Aksjomaty ciała przemiennego}} 
\begin{enumerate}[label=(\arabic*)]
    \item \textbf{Przemienność dodawania}
    \((\forall a,b\in\R)\ a+b=b+a\).
    \item \textbf{Łączność dodawania}
    \((\forall a,b,c\in\R)\ a+(b+c)=(a+b)+c\).
    \item \textbf{Charakteryzacja zera}
    \((\exists 0\in\R)(\forall a\in\R)\ a+0=a\).
    \item \textbf{Istnienie elementów przeciwnych}
    \((\forall a\in\R)(\exists -a\in\R)\ a+(-a)=0\).
    \item \textbf{Przemienność mnożenia}
    \((\forall a,b\in\R)\ a\cdot b=b\cdot a\).
    \item  \textbf{Łączność mnożenia}
    \((\forall a,b,c\in\R)\ a\cdot(b\cdot c)=(a\cdot b)\cdot c\).
    \item \textbf{Charakteryzacja jedynki}
    \((\exists 1\in\R)(\forall a\in\R)\ a\cdot 1=a\).
    \item \textbf{Istnienie elementów odwrotnych}
    \((\forall a\in\R)(\exists a^{-1}\in\R)\ a\cdot a^{-1}=1\).
    \item \textbf{Rozdzielność mnożenia względem dodawania}
    \((\forall a,b,c\in\R)\ a\cdot(b+c)=a\cdot b + a\cdot c\).
\end{enumerate}

\subsection*{{\color{darkred} Aksjomaty porządku}} 
\begin{enumerate}[label=(\arabic*)]
    \item \textbf{Prawo trichotomii}
    \footnote{Aksjomat ten uwzględniamy, gdy przyjmujemy $(\R,+,\cdot,<)$,
    na wykładzie natomiast przyjęliśmy $(\R,+,\cdot,\le)$.}
    \((\forall a,b\in\R)\) zachodzi \emph{dokładnie jedna} z trzech możliwości:
    \begin{equation*}
        a<b,\quad a=b,\quad b<a.
    \end{equation*}
    \item \textbf{Przechodniość}
    \((\forall a,b,c\in\R)(a\le b\land b\le c\imp a\le c)\).
    \item \textbf{Związki nierówności z działaniami}
    \begin{enumerate}
        \item \((\forall a,b,c\in\R)(a\le b\imp a+c\le b+c)\);
        \item \((\forall a,b,c\in\R)(a\le b\land 0\le c\imp ac\le bc)\).
    \end{enumerate}
\end{enumerate} \smallskip

\begin{dedekind}
    \textit{Każdy niepusty, ograniczony z góry podzbiór} $A\subset\R$
    \textit{ma kres górny} $M=\sup{A}\in\R$.
\end{dedekind} \pagebreak

\begin{twier}{Twierdzenie 4.1: Aksjomat Archimedesa}
    Dla każdej pary dodatnich liczb rzeczywistych $a$ i $b$ istnieje liczba
    naturalna $n$, taka że $a<nb$.
\begin{equation*}
    (\forall a,b\in\R_+)(\exists n\in\N)\ a<nb
\end{equation*}
\end{twier}

Twierdzenie to, choć bywa tak zwyczajowo nazywane,
na prawdę aksjomatem w arytmetyce nie jest,
bo wynika z innych aksjomatów teorii liczb rzeczywistych. 

\begin{defr}{Definicja 4.1: Przekrój Dedekinda}
Podział zbioru liczb wymiernych na parę zbiorów $\op{A,B}$, spełniające
warunki\footnote{Podane warunki różnią się swoją postacią,
tym co były przedstawione na wykładzie, niemniej jednak są im równoważne.}:
\begin{enumerate}[label=(\arabic*)]
    \item $A\neq \emptyset \land B\neq \emptyset$,
    \item $A\cup B=\Q$,
    \item $A\cap B=\emptyset$
    \item $(\forall a\in A)(\forall b\in B)\ a<b$.
\end{enumerate}
nazywamy \textbf{przekrojem Dedekinda} zbioru $\Q$. Zbiór $A$
nazywany jest \textbf{klasą dolną} przekroju, a zbiór $B$
\textbf{klasą górną}.
\end{defr}

Przekrój Dedekinda $\op{A,B}$ zdefiniowany w taki sposób może mieć jedną
z trzech następujących postaci, w której: \smallskip
\begin{enumerate}[nosep]
    \item w zbiorze $A$ istnieje element największy,
    \item w zbiorze $B$ istnieje element najmniejszy,
    \item w zbiorze $A$ nie istnieje element największy i
    w zbiorze $B$ nie istnieje element najmniejszy.
\end{enumerate} \smallskip
W trzecim przypadku przekrój wyznacza tzw. \textit{lukę}. 
Aksjomat ciągłości w ujęciu przekrojowym, mówi o tym, że żaden
z przekrojów Dedekinda zbioru $\R$ nie wyznacza luki.\smallskip

Przekroje typu 1 i 2 nazywamy \textit{liczbami rzeczywistymi wymiernymi}.
Dwa przekroje typu 1 i 2 mogą wyznaczać tę samą liczbę wymierną. Relację
równoważności przekrojów zdefiniujemy poniżej. Natomiast przekrój
$\op{A,B}$ wyznaczający lukę nazywamy \textit{liczbą rzeczywistą
niewymierną}. \bigskip

Zdefiniujmy relację równoważności $R$ przekrojów Dedekinda:
\begin{equation*}
    \op{A_1,B_2}\ R\ \op{A_2,B_2}\ifff A_1=A_2\ \lor\ \exists\max{A_1},
    \min{B_2}(\max{A_1}=\min{B_2})\ \lor\ \exists\max{A_2},\min{B_1}
    (\max{A_2}=\min{B_1}).
\end{equation*}



Nasze rozważania doprowadzają nas do \textit{\textbf{{\color{darkred} konstrukcji
zbioru liczb rzeczywistych za pomocą przekrojów Dedekinda}}}\footnote{
W tym miejscu notatki z liczb rzeczywistych na razie zakańczam. Dalsze
wyprowadzenia operacji na liczbach rzeczywistych odkładam na czas bliższy
terminowi egzaminu ustnego. Na kolosie zagadnienia te raczej nie będą potrzebne.
}.
\begin{equation*}
    \R:=\ilorazowy{\set{\op{A,B}:\text{przekroje Dedekinda}}}
\end{equation*}
\newpage


%%%%%%%%%%%%%%%%%%%%%%%%%%%%%%%%%%%%%%%%%%%%%%%%%%%%%%%%%%%%%%%%%%%%%%%%%%%%%%%%%%%%%%%%%%%%%%%%%%%%%%%%%%%%%%%%
                                \section*{Sekcja 5} \smallskip
                                {\Huge\bfseries Ciągi} \bigskip \medskip
%%%%%%%%%%%%%%%%%%%%%%%%%%%%%%%%%%%%%%%%%%%%%%%%%%%%%%%%%%%%%%%%%%%%%%%%%%%%%%%%%%%%%%%%%%%%%%%%%%%%%%%%%%%%%%%%

\begin{defr}{Definicja 5.1: Ciąg nieskończony}
    \textbf{Ciągiem} (nieskończonym) o elementach w zbiorze \(A\)
    nazywamy dowolną funkcję $a:\N\to A$ (ozn. $a_n$, $\set{a_n}_{n\in\N}$
    i $(a_n)_{n\in\N}$).
\end{defr}

\begin{defr}{Definicja 5.2: Podciąg}
    Jeżeli $\seq{a}$ jest ciągiem o elementach w zbiorze $A$ oraz
    $\seq{k}$ jest ciągiem ściśle rosnącym o elementach w $\N$,
    to ciąg $a\circ k=\set{a_{k_n}}_{n\in\N}$ nazywamy \textbf{podciągiem}
    ciągu $\seq{a}$.
\end{defr}

\begin{defr}{Definicja 5.3: Granica ciągu}
    Mówimy, że $g\in \R$ jest granicą ciągu $\seq{a}$, jeżeli
    \begin{equation*}
        \forall_{\varepsilon>0}\ \exists_{n_\varepsilon\in\N}\ 
        \forall_{n\ge n_\varepsilon, n\in\N}\hquad
        |a_n-g|<\varepsilon.
    \end{equation*}
\end{defr}

Gdy $g$ jest granicą ciągu $\seq{a}$, to piszemy, że $\lin{a_n}=g$
lub $a_n\arn g$.

\begin{twier}{Twierdzenie 5.1}
    Ciąg może mieć co najwyżej jedną granicę.
\end{twier}

\begin{defr}{Definicja 5.4: Ciąg Cauchy'ego}
    \textbf{Ciągiem Cauchy'ego} nazywamy ciąg $\seq{a}$, jeśli
    \begin{equation*}
        \forall_{\varepsilon>0}\ \exists_{M\in\N}\
        \forall_{n,m\geq M}\ \forall_{m,n\in\N}\hquad
        |a_n-a_m|<\varepsilon.
    \end{equation*}
\end{defr}

\begin{twier}{Twierdzenie 5.2}
    Każdy ciąg liczb rzeczywistych jest zbieżny wtedy i tylko wtedy, gdy spełnia warunek Cauchy'ego (tzn. jest ciągiem Cauchy'ego).
\end{twier}

\begin{twier}{Twierdzenie 5.3}
    Każdy ciąg zbieżny jest ograniczony, zarówno z dołu, jak i z góry.
\end{twier}

\begin{twier}{Twierdzenie 5.4}
    Jeżeli ciąg $\seq{a}$ jest rozbieżny do $+\infty$ ($-\infty$),
    to jest ograniczony z dołu (góry).
\end{twier}

\begin{twier}{Twierdzenie 5.5}
Jeśli ciąg $\seq{a}$ jest zbieżny do $g$ to każdy podciąg ciągu $\set{a_n}$ też
jest zbieżny do $g$.
\end{twier}

\begin{twier}{Twierdzenia o „arytmetyce” granic 5.6}
    Niech $\seq{a}$ i $\seq{b}$ będą ciągami liczbowymi oraz $a,b \in\R$.
    \begin{enumerate}[label=(\arabic*)]
        \item $a_n\arn a \implies |a_n|\arn|a|$.
        \item $a_n\arn a \implies \sqrt{a_n}\arn\sqrt{a}$.
        \item $a_n\arn a\ \land\  b_n\arn b \implies a_n+b_n\arn a+b$.
        \item $a_n\arn a\ \land\ b_n\arn b\implies a_n-b_n\arn a-b$.
        \item $a_n\arn+\infty\ \land\ \left[b_n\right. \implies a_n+b_n\arn +\infty$.
        \item $a_n\arn-\infty\ \land\ \left.b_n\right] \implies a_n+b_n\arn -\infty$.
        \item $a_n\arn a\ \land\ b_n\arn b\implies a_n b_n \arn ab$.
        \item $a_n\arn 0\ \land\ [b_n]\implies a_nb_n\arn 0$.
        \item $a_n\arn+\infty\ \land\ (\exists q>0)\hquad dddn\hquad b_n>q\implies a_nb_n\arn+\infty$.
        \item $a_n\arn+\infty\ \land\ (\exists q<0)\hquad dddn\hquad b_n<q\implies a_nb_n\arn-\infty$.
        \item $a_n\arn a\ \land\ b_n\arn b\neq0\ \land\ (\forall n\in\N)\ b_n\neq0\implies\frac{a_n}{b_n}\arn\frac{a}{b}$.
        \item $[a_n]\ \land\ |b_n|\arn+\infty\implies \frac{a_n}{b_n}\arn0$.
        \item $a_n\arn0\ \land\ (\exists c>0)\hquad dddn\hquad |b_n|>c \implies \frac{a_n}{b_n}\arn0$.
        \item $a_n\arn+\infty\ \land\ [b_n]\ \land\ dddn\hquad b_n>0\implies \frac{a_n}{b_n}\arn+\infty$.
        \item $a_n\arn-\infty\ \land\ [b_n]\ \land\ dddn\hquad b_n>0\implies \frac{a_n}{b_n}\arn-\infty$.
        \item $a_n\arn+\infty\ \land\ [b_n]\ \land\ dddn\hquad b_n<0\implies \frac{a_n}{b_n}\arn-\infty$.
        \item $a_n\arn-\infty\ \land\ [b_n]\ \land\ dddn\hquad b_n<0\implies \frac{a_n}{b_n}\arn+\infty$.
        \item $a_n\arn a\ \land\ b_n\arn b\implies a_n^{b_n}\arn a^b$.
        \item $a_n\arn a>1\ \land\ b_n\arn +\infty\implies a_n^{b_n}\arn +\infty$.
        \item $a_n\arn a\in(0,1)\ \land\ b_n\arn +\infty\implies a_n^{b_n}\arn 0$.
        \item $a_n\arn a>1\ \land\ b_n\arn -\infty\implies a_n^{b_n}\arn 0$.
        \item $a_n\arn a\in(0,1)\ \land\ b_n\arn -\infty\implies a_n^{b_n}\arn +\infty$.
        \item $a_n\arn +\infty\ \land\ b_n\arn b>0\implies a_n^{b_n}\arn +\infty$.
        \item $a_n\arn +\infty\ \land\ b_n\arn b<0\implies a_n^{b_n}\arn 0$.
    \end{enumerate}
\end{twier}

W celu uproszeczenia zapisu, wprowadziłem do powyższych twierdzeń parę (autorskich!) oznaczeń.
\begin{align*}
    [a_n\quad &ozn. &(\exists m\in\R)(\forall n\in\N)\ &a_n\ge m\\
    a_n]\quad &ozn. &(\exists M\in\R)(\forall n\in\N)\ &a_n\le M\\
    [a_n]\quad &ozn. &(\exists M\in\R)(\forall n\in\N)\ &|a_n|\le M\\
    %)a_n\quad &ozn. &(\forall m\in\R)(\exists N\in\N)\ &a_N<m\\
    %a_n(\quad &ozn. &(\forall M\in\R)(\exists N\in\N)\ &a_N>M\\
    %)a_n(\quad &ozn. &(\forall M\in\R)(\exists N\in\N)\ &|a_N|>M\\
    %[a_n(\quad &ozn. &&[a_n\ \land\ a_n(\\
    %&itd.
\end{align*}

Powyższe twierdzenia, jak i jakiekolwiek inne twierdzenia o arytmetycznych własnościach
granic nie mają zastosowania w przypadku tzw. \textit{wyrażeń nieoznaczonych}.
Do opisania takich wyrażeń wykorzystuje się następujące symbole:
\begin{equation*}
\infty -\infty  ,\hquad 0\cdot\infty,\hquad \frac{0}{0},\hquad \frac{\infty}{\infty},
\hquad 1^{\infty},\hquad \infty^{0},\hquad 0^0.  
\end{equation*}

\begin{twier}{Twierdzenie 5.7}
    Jeżeli ciąg $\seq{a}$ jest zbieżny, a $\seq{b}$ nie, to
    ciąg $\seq{c}$ postaci $c_n=a_n+b_n$ również \underline{nie} jest zbieżny.
\end{twier}

Przydatną własnością wynikającą z tw. 5.7 i tw. 5.6 jest to, że jeśli ciąg $b_n=a_n-g\arn 0$, to $(a_n)$ jest zbieżny i to
dokładnie do granicy $g$.

\begin{twier}{Twierdzenie o szacowaniu granic 5.8}
Załózmy, że $\seq{a}$ i $\seq{b}$ są zbieżnymi ciągami liczb rzeczywistych
oraz $x\in\R$. Zachodzą wówczas następujące implikacje:
\begin{enumerate}[label=(\roman*)]
    \item $\lim\limits_{n\to\infty}{a_n}>x\implies dddn\hquad a_n>x$,
    \item $\lim\limits_{n\to\infty}{a_n}<x\implies dddn\hquad a_n<x$,
    \item $\lim\limits_{n\to\infty}{a_n}>\lim\limits_{n\to\infty}{b_n}\implies dddn\hquad a_n>b_n$,
    \item $dddn\hquad a_n\le b_n\implies \lim\limits_{n\to\infty}{a_n}\le\lim\limits_{n\to\infty}{b_n}$.
\end{enumerate}
\end{twier}

Na wykładzie przedstawiono tylko (iv). (iii) jest równoważne (iv) z prawa kontrapozycji. (i) da się łatwo wywieść z (iii),
wystarczy bowiem przyjąć, że $b_n$ jest ciągiem stałym stale równym $x$. (ii) dowodzimy analogicznie.

\begin{twier}{Twierdzenie o dwóch ciągach 5.9}
Niech $\seq{a}$ i $\seq{b}$ będą ciągami liczb rzeczywistych, wówczas zachodzą następujące implikacje:
\begin{enumerate}[label=\Roman*.]
    \item $\lin{a_n}=+\infty\ \land\ dddn\hquad b_n\ge a_n\implies \lin{b_n}=+\infty$,
    \item $\lin{a_n}=-\infty\ \land\ dddn\hquad b_n\le a_n\implies \lin{b_n}=-\infty$.
\end{enumerate}
\end{twier}

\begin{twier}{Twierdzenie o trzech ciągach 5.10}
Jeżeli $\seq{a}$, $\seq{b}$, $\seq{c}$ są ciągami liczb rzeczywistych, to wówczas
\begin{equation*}
    \lin{a_n}=\lin{c_n}=g\ \land\ dddn\hquad a_n\le b_n\le c_n\implies \lin{b_n}=g.
\end{equation*}
\end{twier}

\begin{twier}{Kryterium d'Alamberta dla ciągów 5.11}
Niech $\seq{a}$ i $\seq{b}$ będą ciągami liczb rzeczywistych, wówczas zachodzą następujące implikacje:
\begin{enumerate}[label=\Roman*.]
    \item $dddn\hquad a_n>0\ \land\ (\exists q>1)\hquad dddn\hquad \frac{a_{n+1}}{a_n}>q\implies \lin{a_n}=+\infty$,
    \item $dddn\hquad a_n>0\ \land\ (\exists q<1)\hquad dddn\hquad \frac{a_{n+1}}{a_n}<q\implies \lin{a_n}=0$.
\end{enumerate}
\end{twier}

\smallskip

\begin{twier}{Twierdzenie o ciągu monotonicznym i ograniczonym 5.12}
\begin{enumerate}[label=\Roman*.]
    \item Każdy niemalejący i ograniczony z góry ciąg $\seq{a}$ jest zbieżny do
    \[\sup\limits_{n\in\N}{a_n}=\sup{\set{a_n:n\in\N}}.\]
    \item Każdy nierosnący i ograniczony z dołu ciąg $\seq{a}$ jest zbieżny do
    \[\inf\limits_{n\in\N}{a_n}=\inf{\set{a_n:n\in\N}}.\]
\end{enumerate}
\end{twier}

\begin{twier}{Twierdzenie Bolzano-Weierstrassa 5.13}
Jeżeli ciąg $\seq{a}\subset \R$ jest ograniczony (zarówno z góry jak i z dołu), to posiada podciąg zbieżny.
\end{twier}

\begin{defr}{Definicja 5.5: Liczba Eulera}
    \textbf{Liczbą Eulera} nazywamy niewymierną liczbę, zdefiniowaną jako granicę 
    \begin{equation*}
        e=\lin{\Bigl(1+\frac{1}{n}\Bigl)^n}.
    \end{equation*}
\end{defr}

\begin{twier}{Twierdzenie 5.14}
Ciąg $e_n=(1+ 1/n)^n$ jest niemalejący, a $\tilde{e}_n=(1+1/n)^{n+1}$ nierosnący. Ponadto
\begin{equation*}
    (\forall n\in\N_+)\hquad \Bigl(1+\frac{1}{n}\Bigl)^n < e < \Bigl(1+\frac{1}{n}\Bigl)^{n+1}.
\end{equation*}
\end{twier}

\textit{Parę przykładów granic ciągów:}
\begin{align}
    \lin{\frac{q^n}{n^k}}=\ &
        \begin{cases}
            +\infty, &\text{dla }\ q>1 \lor(q=1\land k<0)\\
            1, &\text{dla }\ q=1\land k=0\\
            0, &\text{dla }\ q<1\lor(q=1\land k>0)
        \end{cases}\text{, gdzie }q\ge0.\\
    \lin{\sqrt[n]{n}}=\ &1\\
    \lin{\Bigl(1+\frac{1}{n} \Bigl)^{n+1}}=\ &e\\
    \lin{\sum_{k=0}^{n}\frac{1}{k!}}=\ &e
\end{align}\smallskip

\begin{twier}{Twierdzenie 5.15}
    Jeżeli $\seq{a}$ i $\seq{b}$ są ciągami liczb rzeczywistych, to wówczas
    \begin{equation*}
        \lin{a_n}=0\ \land\ \lin{a_n b_n}=g\implies \lin{(1+a_n)^{b_n}}=e^g.
    \end{equation*}
\end{twier}

\begin{twier}{Uogólniona nierówność Bernoulliego 5.16}
    Klasyczną nierówność Bernoulliego można uogólnić do poniższej postaci.
\begin{align*}
    (1+x)^r\ge1+rx,\quad&\text{dla}\hquad x>-1\ \land\ r\ge1\\\
    (1+x)^r\le1+rx,\quad&\text{dla}\hquad x>-1\ \land\ r\in(0,1]
\end{align*}
\end{twier}

\begin{twier}{Twierdzenie 5.17}
Niech $\seq{a}$ będzie ciagiem liczb rzeczywistych i $c\in\R_{+}\setminus\set{1}$ oraz $g\in\R$, wówczas
\begin{enumerate}[label=(\arabic*)]
    \item $\lin{a_n}=+\infty\implies\lin{a_n\log_{c}{\bigl(1+\frac{1}{a_n}\bigl)}}=\frac{1}{\ln{c}},$
    \item $\lin{a_n}=0\implies\lin{\frac{\log_c{(1+a_n)}}{a_n}}=\frac{1}{\ln{c}}$,
    \item $\lin{a_n}=g\implies\lin{log_{c}{a_n}}=\log_{c}{g}$,
    \item $\lin{a_n}=g\implies\lin{\frac{\log_{c}{a_n}-\log_{c}{g}}{a_n-g}}=\frac{1}{g\ln{c}}$.
\end{enumerate}
\end{twier}

\begin{twier}{Szacowanie funkcji eksponencjalnej i logarytmicznej 5.18}
    \begin{enumerate}[label=\Roman*.]
        \item $(\forall x\in\R)\ 1+x\le e^x\ \land\ (\forall x<1)\ e^x\le \frac{1}{1-x}$
        \item $(\forall x>-1)\ \frac{1}{1+x}\le \ln{(1+x)}\le x$
    \end{enumerate}
\end{twier}

Uwaga! Pierwsza z nierówności I. została udowodniona tylko dla $x\ge -1$, jednak można
dokonać rozszerzenia jej stosowalności. Szacowania II. wcale nie pojawiły się na wykładzie, aczkolwiek uznałem je
za przydatne. Dowody powyższych własności można znaleźć 
\href{https://drive.google.com/file/d/1cHNY7oU4XpRmYcPfO8lCgPCpdmvKQY5q/view?usp=sharing}{{\color{blue}\underline{tutaj}}}
na siódmej i dziesiątej stronie.

\begin{twier}{Twierdzenie 5.19}
Niech $\seq{a}$ będzie ciągiem liczb rzeczywistych i $c>0$, wówczas
\begin{enumerate}[label=(\arabic*)]
    \item $\lin{a_n}=0\ \land\ \forall_{n\in\N}\ a_n\neq0\implies\lin{\frac{e^{a_n}-1}{a_n}}=1$,
    \item $\lin{a_n}=0\ \land\ \forall_{n\in\N}\ a_n\neq0\implies\lin{\frac{c^{a_n}-1}{a_n}}=\ln{c}$,
    \item $\lin{a_n}=0\ \land\ \forall_{n\in\N}\ a_n\neq0\ \land\ p\in\R\implies\lin{\frac{(1+a_n)^p-1}{a_n}}=p$,
    \item $\lin{a_n}=+\infty\ \land\ p>0\implies \lin{\frac{ln{a_n}}{a_n^p}}=0$.
\end{enumerate}
\end{twier}

Na wykładzie własności (1) - (3) zostały udowodnione tylko dla ciągów o wyrazach dodatnich, jednakże są one stosowalne
również w przypadku ciągów o wyrazach ujemnych.

\begin{twier}{Twierdzenie Stolza 5.20}
Załóżmy, że ciąg $\seq{b}\subset\R$ jest ściśle monotoniczny oraz $(\forall n\in\N)\ b_n\neq0$. Jeśli $\seq{a}\subset\R$
i istnieje granica
\begin{equation*}
    \lin{\frac{a_{n+1}-a_{n}}{b_{n+1}-b_{n}}}=g,
\end{equation*}
a ponadto zachodzi jeden z następujących warunków:
\begin{enumerate}[label=(\roman*)]
    \item $\lin{a_n}=\lin{b_n}=0$,
    \item $\lin{b_n}=+\infty$,
\end{enumerate}
to wówczas ciąg $\frac{a_n}{b_n}$ jest zbieżny, a ponadto $\lin{\frac{a_n}{b_n}}=g$.
\end{twier}

Granica $g$ nie musi być skończona. Na wykładzie pojawił się tylko warunek (ii), dowód
zbieżności w przypadku (i) można odszukać w
\href{https://www.mimuw.edu.pl/~pawelst/analiza/Analiza_Matematyczna_1/Notatki_itp./Archiwum_files/skryptAM1-2010-11-ver01.010c.pdf}{{\color{blue}\underline{skrypcie Strzeleckiego}}}
(na str. 33) lub na
\href{https://en.wikipedia.org/wiki/Stolz%E2%80%93Ces%C3%A0ro_theorem}{{\color{blue}\underline{angielskiej wiki}}}.

\begin{twier}{Twierdzenie 5.21}
\begin{flalign*}
    c_1,\dots,c_n\in\R\ \land\ d_1,\dots,d_n>0\implies\frac{c_1+\cdots+c_n}{d_1+\cdots+d_n}&\le\max_{k\in\set{1,\dots,n}}{\frac{c_k}{d_k}}\\
    &\ge\min_{k\in\set{1,\dots,n}}{\frac{c_k}{d_k}}
\end{flalign*}
\end{twier}
\newpage


%%%%%%%%%%%%%%%%%%%%%%%%%%%%%%%%%%%%%%%%%%%%%%%%%%%%%%%%%%%%%%%%%%%%%%%%%%%%%%%%%%%%%%%%%%%%%%%%%%%%%%%%%%%%%%%%
                                \section*{Sekcja 6} \smallskip
                                {\Huge\bfseries Przestrzenie metryczne} \bigskip \medskip
%%%%%%%%%%%%%%%%%%%%%%%%%%%%%%%%%%%%%%%%%%%%%%%%%%%%%%%%%%%%%%%%%%%%%%%%%%%%%%%%%%%%%%%%%%%%%%%%%%%%%%%%%%%%%%%%

W sekcji tej $X$ będzie oznaczać dowolny niepusty zbiór,
chyba że w danym ustępie zaznaczę, że jest inaczej. Również pisząc o kuli, będę miał na myśli kulę otwartą.
Wszelkie niezdefiniowane $n$ będzie w domyśle liczbą naturalną.

\begin{defr}{Definicja 6.1: Metryka i przestrzeń metryczna}
    \textbf{Metryką} na zbiorze $X$ nazywa się funkcję $d:X\times X\ar\R$, spełniającą następujące warunki:
    \begin{enumerate}[label=(\arabic*)]
        \item $\forall_{x,y\in X}\hquad d(x,y)\ge 0$, 
        \item $\forall_{x,y\in X}\hquad d(x,y)=d(y,x)$,
        \item $\forall_{x,y\in X}\hquad d(x,y)=0\ifff x=y$,
        \item $\forall_{x,y,z\in X}\hquad d(x,y)+d(y,z)\ge d(x,z)$.
    \end{enumerate}
    Parę $(X,d)$, czyli zbiór $X$ z wyróżnioną metryką $d$ nazywamy \textbf{przestrzenią metryczną}.
\end{defr}

Warunek (1) jest tak na prawdę zbędny, bowiem wynika on z trzech pozostałych:
$$0\stackrel{(3)}{=}d(x,x)\stackrel{(4)}{\le} d(x,y)+d(y,x)\stackrel{(2)}{=}2d(x,y)\implies 0\le d(x,y).$$

\begin{center}
    \subsection*{Przykłady metryk}\label{metryki}
\end{center} 

\begin{enumerate}[label=\textbf{(\arabic*)}]
    \item \textit{metryka dyskretna}, $\quad d_d:X\footnote{$X$ to dowolny niepusty zbiór.}\times X\ar\set{0,1}$
    \begin{equation*}
        \forall_{x,y\in X}\hquad d_d(x,y)=\begin{cases}1,\hquad\text{gdy }x\neq y\\0,\hquad\text{gdy }x=y\end{cases}
    \end{equation*}
    \item \textit{metryka miejska (taksówkowa)}, $\quad d_1:\R^n\times\R^n\ar\R$
    \begin{equation*}
        \forall_{x,y\in\R^n}\hquad d_1(x,y)=\sum_{k=1}^{n}{|x_k-y_k|}
    \end{equation*}
    \item \textit{metryka euklidesowa}, $\quad d_E:\R^n\times\R^n\ar\R$
    \begin{equation*}
        \forall_{x,y\in\R^n}\hquad d_E(x,y)=\sqrt{\sum_{k=1}^{n}{(x_k-y_k)^2}}
    \end{equation*}
    \item \textit{metryka} $d_p$ $(l_p)$, $\quad d_p:\R^n\times\R^n\ar\R$
    \begin{equation*}
        \forall_{x,y\in\R^n}\ \forall_{p\ge1}\hquad d_p(x,y)=\left(\sum_{k=1}^{n}{|x_k-y_k|^{p}}\right)^{\frac{1}{p}}
    \end{equation*}
    \item \textit{metryka maksimum (Czebyszewa, szachowa)}, $\quad d_{\infty}:\R^n\times\R^n\ar\R$
    \begin{equation*}
        \forall_{x,y\in\R^n}\hquad d_{\infty}(x,y)=\max_{k=\set{1,\dots,n}}{|x_k-y_k|}
    \end{equation*}
    \item \textit{metryka supremum}, $\quad d_{sup}:\mathcal{B}(X,Y)
    \footnote{$\mathcal{B}(X,Y)$ to zbiór funkcji ograniczonych ze zbioru $X$ do przestrzeni metrycznej $(Y,d_y)$.}
    \times\mathcal{B}(X,Y)\ar\R$
    \begin{equation*}
        \forall_{f,g\in\mathcal{B}(X,Y)}\hquad d_{sup}=\sup_{x\in X}{d_y\left(f(x),g(x)\right)}
    \end{equation*}
        \begin{enumerate}
            \item \textit{Przypadek szczególny.}\footnote{Na wykładzie pojawił się tylko przypadek szczególny. Ogólna definicja
            jest moim dodatkiem.}
            Niech $\mathcal{B}(X,\R)$ będzie zbiorem funkcji ograniczonych ze
            zbioru $X$ do zbioru liczb rzeczywistych $\R$, wyposażonego w metrykę euklidesową. Metryka supremum
            przyjmuje wówczas postać
            \begin{equation*}
                \forall_{f,g\in\mathcal{B}(X,\R)}\hquad d_{sup}=\sup_{x\in X}{|f(x)-g(x)|}.
            \end{equation*}
        \end{enumerate}
    \item \textit{metryka rzymska}, $\quad d_r:\R^2\times\R^2\ar\R$
    \begin{equation*}
        \forall_{x,y\in\R^2}\hquad d_r(x,y)=\begin{cases}
            \sqrt{(x_1-y_1)^2+(x_2-y_2)^2},\hquad\text{gdy punkty $x$, $y$ i $(0,0)$ leżą na jednej prostej}\\
            \sqrt{x_1^2+x^2_2}+\sqrt{y_1^2+y_2^2},\hquad\text{gdy punkty $x$, $y$ i $(0,0)$ nie leżą na jednej prostej}
        \end{cases}
    \end{equation*}
\end{enumerate}
Przestrzeń $(\R^n,d_p)$ bywa oznaczana jako $\ell^p_n$.
\medskip

\begin{defr}{Definicja 6.2: Kula otwarta}
    \textbf{Kulą otwartą} w przestrzeni metrycznej $(X,d)$ o środku w punkcie $a\in X$ i promieniu $r>0$
    nazywamy zbiór
    \begin{equation*}
        B(a,r)=\set{x\in X:d(a,x)<r}.
    \end{equation*}
\end{defr}

\begin{defr}{Definicja 6.3: Otoczenie}
    \textbf{Otoczeniem} punktu $a$ w przestrzeni metrycznej $(X,d)$ nazywamy zbiór $Y\subset X$, jeżeli
    \begin{equation*}
        \exists_{r>0}\hquad B(a,r)\subset Y.
    \end{equation*}
\end{defr}

\begin{defr}{Definicja 6.4: Zbieżność w przestrzeni metrycznej}
    Mówimy, że ciąg $\seq{x}\subset X$ jest zbieżny do $g$ w przestrzeni metrycznej $(X,d)$ wtedy i tylko wtedy, gdy
    \begin{enumerate}[label=(\arabic*)]
        \item $\lin{d(x_n,g)}=0$,
        \item $\forall_{\varepsilon>0}\ \exists_{N>0}\ \forall_{n>N}\hquad d(x_n,g)<\varepsilon$,
        \item $\forall_{\varepsilon>0}\ \exists_{N>0}\ \forall_{n>N}\hquad x_n\in B(g,\varepsilon)$,
        \item dla dowolnego otoczenia $U$ punktu $g$ tylko skończona liczba elementów ciągu $\seq{x}$
        leży poza $U$ (prawie wszystkie $x_n\in U$).
    \end{enumerate}
\end{defr}

Powyższe warunki są sobie równoważne. Jeżeli ciąg spełnia jeden z nich, to spełnia wszystkie na raz.\smallskip

\textit{Uwaga! W ogólnym przypadku (dla ogólnej przesterzeni metrycznej $(X,d)$) nie ma czegoś
takiego jak rozbieżność do $\pm\infty$}.

\begin{defr}{Definicja 6.5: Ciąg ograniczony i nieograniczony}
    Ciąg $\seq{x}\subset X$ nazywamy \textbf{ograniczonym} w przestrzeni metrycznej $(X,d)$, jeżeli
    \begin{equation*}
        \exists_{a\in X}\ \exists_{r>0}\ \forall_{n\in\N}\hquad x_n\in B(a,r).
    \end{equation*}
    W przeciwnym razie ciąg $\seq{x}$ nazywamy \textbf{nieograniczonym}.
\end{defr}

\begin{defr}{Definicja 6.6: Ciąg Cauchy'ego}
    Ciąg $\seq{x}\subset X$ nazywamy \textbf{ciągiem Cauchy'ego} w przestrzeni metrycznej $(X,d)$, jeśli
    \begin{equation*}
        \forall_{\varepsilon>0}\ \exists_{M\in\N}\ \forall_{n\ge M}\hquad d(x_n,x_m)<\varepsilon.
    \end{equation*}
\end{defr}

Należy zwrócić uwagę, że istnieją przestrzenie metryczne, w których nie wszystkie ciągi Cauchy'ego są zbieżne. Natomiast
wszystkie ciągi zbieżne są ciągami Cauchy'ego.

\begin{defr}{Definicja 6.7: Przestrzeń zupełna}
    Przestrzeń metryczna $(X,d)$ jest \textbf{zupełna}, jeśli każdy ciąg Cauchy'ego w tej przestrzeni jest zbieżny.
\end{defr}

\begin{defr}{Definicja 6.8: Wnętrze}
    \textbf{Wnętrzem} zbioru $A\subset X$ w przestrzeni metrycznej $(X,d)$ nazywamy zbiór punktów, dla których $A$
    jest otoczeniem,
    \begin{equation*}
        \interior{A}=\set{x\in A:\exists_{r>0}\hquad B(x,r)\subset A}.
    \end{equation*}
\end{defr}

\begin{defr}{Definicja 6.9: Zbiór otwarty}
    Zbiór $A\subset X$ nazywamy \textbf{otwartym} w przestrzeni metrycznej $(X,d)$,
    jeżeli jest otoczeniem każdego swojego punktu, czyli
    \begin{equation*}
        \forall_{x\in A}\ \exists_{r>0}\hquad B(x,r)\subset A.
    \end{equation*}
    Zbiór $A$ jest otwarty wtedy i tylko wtedy, gdy $A=\interior{A}$.
\end{defr}

\begin{defr}{Definicja 6.10: Domknięcie i zbiór domknięty}
    \textbf{Domknięciem} zbioru $A\subset X$ w przestrzeni metrycznej $(X,d)$ nazywamy zbiór
    \begin{equation*}
        \closure{A}=\set{x\in X:\forall_{r>0}\hquad B(x,r)\cap A\neq\emptyset}.
    \end{equation*}
    $A$ jest \textbf{zbiorem domkniętym}, gdy $A=\closure{A}$.
\end{defr}

Domknięcie i wnętrze zbioru $A$ alternatywnie oznaczamy jako $cl\ A$ oraz $\mathring{A}$.

\begin{defr}{Definicja 6.11: Brzeg}
    \textbf{Brzegiem} zbioru $A\subset X$ w przestrzeni metrycznej $(X,d)$ nazywamy zbiór
    \begin{equation*}
        \partial{A}=\closure{A}\setminus\interior{A}.
    \end{equation*}
\end{defr}

\begin{twier}{Twierdzenie 6.1}
    Zbiór $A\subset X$ jest domknięty w przestrzeni metrycznej $(X,d)$ \textit{wtedy i tylko wtedy, gdy} każdy zbieżny ciąg $\seq{a}\subset A$
    posiada granicę $g\in A$.
    \begin{equation*}
        A=\closure{A}\iff\forall_{a\in A^\N}\ \exists_{g\in X}\ \lin{d(a_n,g)=0}\implies g\in A
    \end{equation*}
\end{twier}

\begin{twier}{Twierdzenie 6.2}
    Zbiór $A\subset X$ jest otwarty w przestrzeni metrycznej $(X,d)$ \textit{wtedy i tylko wtedy, gdy} zbiór $X\setminus A$ jest domknięty.
    \begin{equation*}
        A=\interior{A}\iff X\setminus A=\Closure{\left(X\setminus A\right)}
    \end{equation*}
\end{twier}

\begin{twier}{Twierdzenie 6.3}
    \begin{enumerate}[label=(\arabic*)]
        \item Suma dowolnej rodziny zbiorów otwartych jest zbiorem otwartym.
        \item Przecięcie skończonej rodziny zbiorów otwartych jest zbiorem otwartym.
        \item Przecięcie dowolnej rodziny zbiorów domkniętych jest zbiorem domkniętym.
        \item Suma skończonej rodziny zbiorów domkniętych jest zbiorem domkniętym.
    \end{enumerate}
\end{twier}

\begin{defr}{Definicja 6.12: Pokrycie i pokrycie otwarte}
    Rodzinę zbiorów $\ri{U}$, zawartą w przestrzeni $X$, nazywamy \textbf{pokryciem} zbioru $A\subset X$, jeśli
    \begin{equation*}
        A=\bigcup_{i\in I}{U_i}.
    \end{equation*}
    Mówimy, że pokrycie $\ri{U}$ jest \textbf{pokryciem otwartym}, jeśli $\forall_{i\in I}\ U_i=\interior{U_i}$.
\end{defr}

\begin{defr}{Definicja 6.13: Podpokrycie}
    Niech $\ri{U}$, zawarty w przestrzeni $X$, będzie pokryciem zbioru $A$ oraz $K\subset I$. Jeśli $\set{U_k}_{k\in K}$
    jest wówczas pokryciem $A$, to $\set{U_k}_{k\in K}$ nazywamy \textbf{podpokryciem} pokrycia $\ri{U}$.
\end{defr}

\begin{defr}{Definicja 6.14: Zbiór zwarty}
    Zbiór $A$, zawarty w przestrzeni $X$, nazywamy \textbf{zwartym}, jeśli z każdego pokrycia otwartego zbioru $A$ można
    wybrać podpokrycie skończone.
\end{defr}

\begin{twier}{Twierdzenie 6.4}
    Zbiór $A$, zawarty w przestrzeni $X$, jest zwarty \textit{wtedy i tylko wtedy, gdy} zbiór $A$ jest \textit{ciągowo zwarty}, czyli
    gdy z każdego ciągu $\seq{x}\subset A$ można wybrać podciąg $\set{x_{n_k}}_{k\in K\subset\N}$ zbieżny w $A$.
\end{twier}

\begin{twier}{Twierdzenie 6.5}
    Zbiór $A$, zawarty w przestrzeni metrycznej $(\R^n,d_E)$, jest zwarty \textit{wtedy i tylko wtedy, gdy} $A$ jest
    domknięty i ograniczony.
\end{twier}

\begin{defr}{Definicja 6.15: Zbiór gęsty}
    Zbiór $A\subset X$ nazywamy \textbf{gęstym} w przestrzeni metrycznej $(X,d)$, jesli $\closure{A}=X$. 
\end{defr}

\begin{defr}{Definicja 6.16: Zbiór spójny i niespójny}
    Zbiór $A$, zawarty w przestrzeni $X$, nazywamy \textbf{niespójnym}, jeśli 
    \begin{equation*}
        \exists_{B_1,B_2\neq\emptyset}\ \left(B_1\cup B_2=A\ \land\ \closure{B_1}\cap B_2=\emptyset\ \land\ B_1\cap\closure{B_2}=\emptyset\right). 
    \end{equation*}
    W przeciwnym przypadku zbiór $A$ jest \textbf{spójny}, czyli gdy
    \begin{equation*}
        \forall_{B_1,B_2\neq\emptyset}\ \left(B_1\cup B_2=A\implies\closure{B_1}\cap B_2\neq\emptyset\ \lor\ B_1\cap\closure{B_2}\neq\emptyset\right).
    \end{equation*}
\end{defr}

\begin{defr}{Definicja 6.17: Równoważność metryk}
    Niech $X$ będzie niepustym zbiorem, a $d_1$ i $d_2$ metrykami na nim. Mówimy, że metryki $d_1$ i $d_2$ są \textbf{równoważne}, gdy
    \begin{enumerate}[label=\Roman*.]
        \item $\forall_{x,y\in X}\ \forall_{\delta>0}\ \exists_{\varepsilon_1,\varepsilon_2>0}\hquad
        (d_1(x,y)<\varepsilon_1\implies d_2(x,y)<\delta)\ \land\ (d_2(x,y)<\varepsilon_2\implies d_1(x,y)<\delta$),
        \item $\forall_{x\in X}\ \forall_{\delta>0}\ \exists_{\varepsilon_1,\varepsilon_2>0}\hquad
        B_1(x,\varepsilon_1)\subset B_2(x,\delta)\ \land\ B_2(x,\varepsilon_2)\subset B_1(x,\delta)$.
    \end{enumerate}
\end{defr}
Wymienione wyżej warunki I. i II. są równoważne, więc wystarczy wykazanie tylko jednego z nich.

\begin{twier}{Twierdzenie 6.6}
    Metryki $d_p$ i $d_\infty$ są sobie równoważne.\footnote{Patrz \textit{\nameref{metryki}.}}
\end{twier}

Jeżeli każda metryka $d_p$ jest równoważna z każdą metryką $d_\infty$, to z przechodniości relacji równoważności,
wszystkie metryki $d_p$ są sobie równoważne. Przykładowo metryka miejska ($d_1$) jest równoważna z euklidesową ($d_2$). 

\begin{twier}{Twierdzenie 6.7}
    Dane są przestrzenie metryczne $(X,d_1)$ i $(X,d_2)$. Metryki $d_1$ i $d_2$ są \textit{równoważne}, jeśli
    \begin{equation*}
        \forall_{\seq{x}\subset X}\hquad \lin{d_1(x_n,g)}=0\iff \lin{d_2(x_n,g)}=0,
    \end{equation*}
    tzn. dowolny ciąg $\seq{x}$ o wyrazach z $X$ jest zbieżny do $g$ w $(X,d_1)$ wtedy i tylko wtedy, gdy jest zbiezny do
    $g$ w $(X,d_2)$.
\end{twier}

\begin{twier}{Twierdzenie 6.8}
    Dana jest przestrzeń $\ell^p_N$, wówczas
    \begin{equation*}
        \forall_{\seq{x}\subset R^N}\hquad \left(\lin{d_p(x_n,g)=0}\iff\forall_{i\in\set{1,\dots,N}}\hquad\lin{x_{n,i}=g_i}\right).
    \end{equation*} 
\end{twier}

Wyrazy postaci $a_i$ oznaczają $i$-tą współrzędną punktu $a=(a_1,\dots,a_N)\in\R^N$.
\newpage


%%%%%%%%%%%%%%%%%%%%%%%%%%%%%%%%%%%%%%%%%%%%%%%%%%%%%%%%%%%%%%%%%%%%%%%%%%%%%%%%%%%%%%%%%%%%%%%%%%%%%%%%%%%%%%%%
                                \section*{Sekcja 7} \smallskip
                                {\Huge\bfseries Granica i ciągłośc funkcji} \bigskip \medskip
%%%%%%%%%%%%%%%%%%%%%%%%%%%%%%%%%%%%%%%%%%%%%%%%%%%%%%%%%%%%%%%%%%%%%%%%%%%%%%%%%%%%%%%%%%%%%%%%%%%%%%%%%%%%%%%%

Definicje i własności w sensie Cauchy'ego będą oznaczane poprzez $(C)$, a Heinego za pomocą $(H)$.

\begin{defr}{Definicja 7.1: Punkt skupienia}
    W przestrzeni metrycznej $(X,d)$ punkt $p\in X$ jest \textbf{punktem skupienia} zbioru $A\subset X$ wtedy i tylko wtedy,
    gdy $p\in\closure{A\setminus\set{p}}$.\smallskip

    Alternatywnie punkt $p$ jest punktem skupienia zbioru $A$ wtedy i tylko wtedy, gdy 
    jest granicą pewnego ciągu elementów zbiorów $A\setminus\set{p}$.
\end{defr}

\begin{defr}{Definicja 7.2: Granica funkcji w punkcie}
Dane są przestrzenie metryczne $(X,d_X)$, $(Y,d_Y)$. Niech $A\subset X$, $g\in Y$, $f:A\to Y$ i $p\in\closure{A}$.
Mówimy, że funkcja $f$ ma \textbf{granicę} $g$ \textbf{w punkcie} (skupienia) $p$, jeśli
\begin{enumerate}
    \item[$(C)$] $\begin{aligned}[t]
\forall_{\varepsilon>0}\ \exists_{\delta>0}\ \forall_{x\in A\setminus\set{p}}\hquad&d_X(p,x)<\delta\implies d_Y(f(x),g)<\varepsilon,\\
                &x\in B_X(p,\delta)\implies f(x)\in B_Y(g,\varepsilon),
\end{aligned}$
    \item[$(H)$] $\forall_{\seq{x}\,\subset\,A\setminus\set{p}}\hquad x_n\arr{d_X}p\implies f(x_n)\arr{d_Y}g$.
\end{enumerate}
\end{defr}

Wyrażenie $x_n\arr{d_X}p$ oznacza zbieżność $x_n$ do $p$ w przestrzeni metrycznej $(X,d_X)$. Wprowadziłem pojęcie punktu
skupienia, które się nie pojawiło na wykładzie, by móc uogólnić definicję granicy funkcji w punkcie dla przypadku, gdy $D_f$
jest podzbiorem $X$. 

\begin{defr}{Definicja 7.3: Granica jednostronna funkcji w punkcie}
Dana jest przestrzeń metryczna $(Y,d_Y)$, zbiór $X\subset\R$ oraz funkcja $f:X\to Y$.
Mówimy, że $g\in Y$ jest \textbf{granicą lewostronną
(prawostronną)} funkcji $f$ w punkcie (skupienia) $a\in\closure{X}$, jeżeli
\begin{enumerate}
    \item[(C)] $\forall_{\varepsilon>0}\ \exists_{\delta>0}\ \forall_{x\in(a-\delta,\ a)\cap X}\hquad f(x)\in B_Y(g, \varepsilon)\quad
    \left(\forall_{\varepsilon>0}\ \exists_{\delta>0}\ \forall_{x\in(a,\ a+\delta)}\hquad f(x)\in B_Y(g, \varepsilon)\right)$,
    \item[(H)] $\forall_{\seq{x}\,\subset X\setminus[a,\,+\infty)}\hquad x_n\arr{}p\implies f(x_n)\arr{d_Y}g\quad
    \left(\forall_{\seq{x}\,\subset X\setminus(-\infty,\,a]}\hquad x_n\arr{}p\implies f(x_n)\arr{d_Y}g\right)$.
\end{enumerate}
\end{defr}

\begin{defr}{Definicja 7.4: Granica funkcji w $\boldsymbol{+\infty}$ i $\boldsymbol{-\infty}$}
Dana jest przestrzeń metryczna $(Y, d_Y)$ oraz funkcja $f:\R\supset X\to Y$. Mówimy, że $g\in Y$ jest \textbf{granicą funkcji}
$f$ \textbf{w} $\boldsymbol{+\infty}$ $\boldsymbol{(-\infty)}$, jeżeli
\begin{enumerate}
    \item[(C)] $\forall_{\varepsilon>0}\ \exists_{r\in\R}\ \forall_{x\in X\setminus(-\infty,\, r]}\hquad f(x)\in B_Y(g,\varepsilon)
    \quad \left(\forall_{\varepsilon>0}\ \exists_{r\in\R}\ \forall_{x\in X\setminus[r,\, +\infty)}\hquad f(x)\in B_Y(g,\varepsilon)\right)$,
    \item[(H)] $\forall_{\seq{x}\,\subset X}\hquad x_n\arr{}+\infty\implies f(x_n)\arr{d_Y}g\quad
    \left(\forall_{\seq{x}\,\subset X}\hquad x_n\arr{}-\infty\implies f(x_n)\arr{d_Y}g\right)$.
\end{enumerate}
\end{defr}

\begin{defr}{Definicja 7.5: Zbieżność od dołu/góry}
Dana jest przestrzeń metryczna $(X,d_X)$ oraz funkcja $f:X\supset A\to\R$. Mówimy, że funkcja $f$ zbiega do granicy $g\in\R$
w punkcie $a\in\closure{A}$ \textbf{od dołu (od góry)}, jeżeli
    \begin{enumerate}
        \item[(C)] $\begin{aligned}[t]
        \forall_{\varepsilon>0}\ \exists_{\delta>0}\ \forall_{x\in B_X(a, \delta)\cap A\setminus\set{p}}\hquad& f(x)\in(g-\varepsilon,g),\\
        &\left(f(x)\in(g,g+\varepsilon)\right),
        \end{aligned}$ 
        \item[(H)] $\begin{aligned}[t]
        \forall_{\seq{x}\subset X\setminus\set{p}}\hquad x_n\arr{d_X}a\implies f(x_n)\arr{}g\ \land\ dddn\ \ &f(x_n)<g,\\
        &\left(f(x_n)>g\right).
        \end{aligned}$
    \end{enumerate}
\end{defr}

\begin{defr}{Definicja 7.6: Granica niewłaściwa w punkcie}
Dana jest przestrzeń metryczna $(X,d_X)$ oraz funkcja $f:X\supset A\to\R$. Mówimy, że funkcja $f$ \textbf{rozbiega do}
$\boldsymbol{+\infty}$ \textbf{($\boldsymbol{-\infty}$) w punkcie} $a\in\closure{A}$, jeśli
    \begin{enumerate}
        \item[(C)] $\begin{aligned}[t]
        \forall_{r\in\R}\ \exists_{\delta>0}\ \forall_{x\in B_X(a, \delta)\cap A\setminus\set{p}}\hquad &f(x)>r,\\
        &\left(f(x)<r\right),
        \end{aligned}$
        \item[(H)] $\forall_{\seq{x}\subset A\setminus\set{p}}\hquad x_n\arr{d_X}a\implies f(x_n)\arr{}+\infty\ (-\infty)$.
    \end{enumerate}
\end{defr}

\bigskip

\textit{\textbf{\underline{Przykłady granic funkcji}}}

\begin{align}
    &\lim_{x\to 0}{x\,\sin{\frac{1}{x}}}&=0\\
    &\lim_{x\to 0}{\frac{\sin{x}}{x}}&=1\\
    &\lim_{x\to 0}{\frac{1-\cos{x}}{x^2}}&=\frac{1}{2}\\
    &\lim_{x\to 0}{\frac{x}{\tan{x}}}&=1\\
    &\lim_{x\to 0}{\frac{e^x-1}{x}}&=1\\
    &\lim_{x\to 1}{\frac{x^p-1}{x-1}}&=p\\
    &\lim_{x\to 0}{\frac{\ln{(1+x)}}{x}}&=1
\end{align}

\bigskip

\begin{defr}{Definicja 7.7: Ciągłość funkcji}
    Dane są przestrzenie metryczne $(X,d_X)$, $(Y, d_Y)$, funkcja $f:X\supset A\to Y$ oraz $a\in A$. Mówimy, że
    funkcja $f$ jest \textbf{ciągła w punkcie} $a$, jeśli $\lim_{x\to a}{f(x)}=f(a)$, czyli
    \begin{enumerate}
        \item[($C$)] $\forall_{\varepsilon>0}\ \exists_{\delta>0}\ \forall_{x\in A}\hquad d_X(x, a)<\delta\implies d_Y(f(x), f(a))<\varepsilon$,
        \item[($H$)] $\forall_{\seq{x}\subset A}\hquad x_n\arr{d_X} a\implies f(x_n)\arr{d_Y}f(a)$.  
    \end{enumerate}
    Funkcję $f$ nazywamy \textbf{ciągłą}, jeśli jest ciągła w każdym punkcie $x\in A$.
\end{defr}

\begin{defr}{Definicja 7.8: Ciągłość jednostajna}
    Dane są przestrzenie metryczne $(X, d_X)$, $(Y, d_Y)$ oraz funkcja $f:X\to Y$. Mówimy, że funkcja $f$ jest \textbf{jednostajnie
    ciągła} na zbiorze $A\subset X$, jeżeli
    \begin{equation*}
        \forall_{\varepsilon>0}\ \exists_{\delta>0}\ \forall_{x_1, x_2\in A}\hquad d_X(x_1, x_2)<\delta\implies d_Y(f(x_1), f(x_2))<\varepsilon.
    \end{equation*}
\end{defr}

\begin{twier}{Twierdzenie 7.1}
    Każda funkcja jednostajnie ciągła jest również ciągła.
\end{twier}

Jednakże implikacja w drugą stronę już nie zachodzi.

\begin{twier}{Twierdzenie 7.2}
    Funkcja ciągła na zbiorze zwartym jest jednostajnie ciągła.
\end{twier}

\begin{twier}{Twierdzenie 7.3}
    Jeśli $(X, d_X)$ jest przestrzenią zupełną, a $f:X\supset A\to Y$ funkcją ciągłą, to wówczas istnieje funkcja ciągła
    $g: \closure{A}\to Y$, taka że $g|_A=f$.
\end{twier}

\begin{twier}{Twierdzenie 7.4}
    Obraz zbioru zwartego pod działaniem funkcji ciągłej jest zbiorem zwartym\footnote{zatem również i domkniętym}.
\end{twier}

Można również powiedzieć, że ciągły obraz zbioru zwartego jest zwarty.

\begin{twier}{Twierdzenie 7.5}
    Niech $(X, d_X)$, $(Y, d_Y)$ będą przestrzeniami metrycznymi, a $f:X\to Y$ funkcją ciągłą, wówczas
    \begin{equation*}
        Y\supset A=\interior{A}\implies f^{-1}(A) = \interior{f^{-1}(A)}.
    \end{equation*}
\end{twier}

\begin{twier}{Twierdzenie o „arytmetyce” funkcji ciągłych 7.6}
\begin{enumerate}[label=(\arabic*)]
\item 
    Jeśli funkcje $f,\ g: X\to Y$ są ciągłe w punkcie $p\in X$, to $f\pm g$, $f\cdot g$ są ciągłe w $p$. Ponadto jeżeli
    $\forall_{x\in X}\ g(x)\neq0$, to $f/g$ jest określona na $X$ i ciągła w $p$.
\item  
    Jeśli funkcja $f:A\to Y$ jest ciągła w $p\in A$, a $g: B\to Y$, gdzie $f(A)\subset B$, jest ciągła w $f(p)$, to złożenie
    $g\circ f$ jest ciągłe w punkcie $p$.
\end{enumerate}
\end{twier}

\begin{twier}{Twierdzenie Weierstrassa o osiąganiu kresów 7.7} %ponoć coś tutaj miałem poprawić, ale nie pamiętam co
Niech $(X, d_X)$ będzie przestrzenią metryczną, a $f:X\to\R$ funkcją ciągłą. Jeśli $A\neq\emptyset$ oraz $A=\closure{A}$, to
$\inf{f(A)},\, \sup{f(A)}\in f(A)$, czyli
\begin{equation*}
    \exists_{a_1, a_2\in A}\quad f(a_1) = \inf{f(A)}\ \land\ f(a_2) = \sup{f(A)}.
\end{equation*}
\end{twier}

\textit{\textbf{\underline{Przypadek szczególny:}}} Jeśli $f:[a,b]\to\R$ jest funkcją ciągłą, to
\begin{equation*}
    \exists_{c, d\in[a,b]}\quad f(c)=\min{\set{f(x):x\in[a,b]}}\ \land\ f(d)=\max{\set{f(x):x\in[a,b]}}.
\end{equation*}

\begin{twier}{Twierdzenie 7.8}
    Ciągły obraz zbioru spójnego jest spójny.
\end{twier}

\begin{twier}{Twierdzenie Darboux o wartości pośredniej 7.9}
    Niech $(X, d_X)$ będzie przestrzenią metryczną, a $f:X\to\R$ funkcją ciągłą. Jeśli $A\subset X$ jest zbiorem spójnym, to
    \begin{equation*}
        \forall_{y_1, y_2\in f(A),\ y_1\le y_2}:\hquad [y_1, y_2]\subset f(A)\text{, czyli }\forall_{y\in R,\ y_1\le y\le y_2}\ \exists_{x\in A}\hquad y=f(x).
    \end{equation*}
\end{twier}

\textit{\textbf{\underline{Przypadek szczególny:}}} Jeśli $f:[a,b]\to\R$ jest funkcją ciągłą, to
\begin{equation*}
    \forall_{y\in[min{\set{f(a),f(b)}},\, max{\set{f(a),f(b)}}]}\ \exists_{x\in[a,b]}\hquad y=f(x).
\end{equation*}

\begin{twier}{Twierdzenie o ciągłości funkcji odwrotnej 7.10}
    Jeżeli funkcja ciągła $f:A\to\R$ jest ściśle monotoniczna na przedziale $A\subset\R$, wówczas
    $f^{-1}:f(A)\to A$ jest również ciągła i ściśle monotoniczna.
\end{twier}

\begin{twier}{Twierdzenie o granicy złożenia 7.11}
    $$\lim_{x\to a}{f(x)}=b\ \land\ \lim_{y\to b}{g(y)=c}\implies \lim_{x\to a}{g(f(x))=c}$$
\end{twier}
\newpage

%%%%%%%%%%%%%%%%%%%%%%%%%%%%%%%%%%%%%%%%%%%%%%%%%%%%%%%%%%%%%%%%%%%%%%%%%%%%%%%%%%%%%%%%%%%%%%%%%%%%%%%%%%%%%%%%
                                \section*{Sekcja 8} \smallskip
                                {\Huge\bfseries Pochodne} \bigskip \medskip
%%%%%%%%%%%%%%%%%%%%%%%%%%%%%%%%%%%%%%%%%%%%%%%%%%%%%%%%%%%%%%%%%%%%%%%%%%%%%%%%%%%%%%%%%%%%%%%%%%%%%%%%%%%%%%%%

\begin{defr}{Definicja 8.1: Pochodna w punkcie}
Niech $f:\R\supset X\to\R$ będzie funkcją określoną w otoczeniu punktu $a$
(tzn. $\exists_{\varepsilon>0}\ (a-\varepsilon, a+\varepsilon)\subset X$).
Jeśli istnieje granica
\begin{equation*}
    \lim_{h\to 0}{\frac{f(a+h)-f(a)}{h}}=\lim_{x\to a}{\frac{f(x)-f(a)}{x-a}},
\end{equation*}
to nazywamy ją \textbf{pochodną} funkcji $f$ \textbf{w punkcie} $a$.
\end{defr}

\begin{defr}{Definicja 8.2: Funkcja pochodna}
Jeżeli funkcja $f:X\to\R$ jest różniczkowalna w każdym punkcie pewnego zbioru $A\subset X$, to funkcję
$f':A\ni a\mapsto f'(a)\in R$ nazywamy \textbf{funkcją pochodną} lub \textbf{pochodną} funkcji $f$.
\end{defr}

\begin{defr}{Definicja 8.3: Jednostronna pochodna w punkcie}
Niech $f:X\to\R$ będzie funkcją określoną w lewostronnym (prawostronnym) otoczeniu punktu $a$
(tzn. $\exists_{\varepsilon>0}\ (a-\varepsilon, a]\ \ ([a,a+\varepsilon))$. Jeśli istnieje granica
\begin{equation*}
    \lim_{h\to 0^{-}}{\frac{f(a+h)-f(a)}{h}}=\lim_{x\to a^{-}}{\frac{f(x)-f(a)}{x-a}}\quad
    \left( \lim_{h\to 0^{+}}{\frac{f(a+h)-f(a)}{h}}=\lim_{x\to a^{+}}{\frac{f(x)-f(a)}{x-a}} \right),
\end{equation*}
to nazywamy ją \textbf{pochodną lewostronną (prawostronną)} funkcji $f$ w punkcie $a$ i oznaczamy jako $f'_{-}(a)$
($f'_{+}(a)$).
\end{defr}

\begin{twier}{Twierdzenie 8.1}
Pochodne jednostronne $f'_{-}(a)$ i $f'_{+}(a)$ funkcji $f$ w punkcie $a$ istnieją oraz są sobie równe 
$f'_{-}(a)=f'_{+}(a)=g\in\R$ \textit{wtedy i tylko wtedy, gdy} pochodna funkcji $f$ w punkcie $a$ istnieje i jest równa $g$.
\begin{equation*}
    \exists_{f'_{-}(a),\ f'_{+}(a)}\hquad f'_{-}(a)=f'_{+}(a)=g\iff\exists_{f'(a)}\hquad f'(a)=g
\end{equation*}
\end{twier}

\begin{twier}{Twierdzenie o arytmetycznych własnościach pochodnej 8.2}
\begin{enumerate}[label=\Roman*.]
    \item Jeśli $f,\, g:\R\supset X\to\R$ są różniczkowalne w $x\in X$, to
    \begin{gather}
        (f+g)'(x)=f'(x)+g'(x)\ ,\\(f\cdot g)'(x)=f'(x)\,g(x)+f(x)\,g'(x)\ .
    \end{gather}
    \item Jeśli $f,\,g:\R\supset X\to\R$ jest różniczkowalna w $x\in X$ oraz $\forall_{x\in X}\ g(x)\neq 0$, to
    \begin{equation*}
        \left(\frac{f}{g}\right)'(x)=\frac{f'(x)\,g(x)-f(x)\,g'(x)}{g(x)^2}\ .
    \end{equation*}
\end{enumerate}
\end{twier}

Pochodna jest operacją liniową

\begin{twier}{Twierdzenie 8.3}
    Jeśli funkcja ma pochodną w punkcie $x$, to jest również w nim ciągła. 
\end{twier}

\begin{twier}{Reguła łańcuchowa 8.4}
Niech $g:\R\supset A\to\R$ i $f:\R\supset B\to\R$, gdzie $B\supset g(A)$.
Załóżmy, że funkcja $g$ jest określona w otoczeniu $a\in A$, a $f$ w otoczeniu $g(a)$.
Jeśli $g$ jest różniczkowalna w punkcie $a$ i $f$ jest różniczkowalna w $g(a)$, to złożenie
$f\circ g$ jest funkcją różniczkowalną w punkcie $a$ i ma pochodną
\begin{equation*}
    (f\circ g)'(a)=f'(g(a))\cdot g(a)\ .
\end{equation*}
\end{twier}

\begin{twier}{Twierdzenie o pochodnej funkcji odwrotnej 8.5}
    Niech $f:\R\supset A\to f(A)\subset \R$ będzie funkcją różnowartościową, różniczkowalną w $a\in A$.
    Jeśli $f'(a)\ne0$ i $f^{-1}$ jest ciągła w $f(a)$, to $f^{-1}$ jest różniczkowalna w $f(a)$ i zachodzi wzór
    \begin{equation*}
        \left( f^{-1} \right)'(f(x))=\frac{1}{f'(x)}.
    \end{equation*}
\end{twier}
\bigskip

Parę pochodnych $n$-tego rzędu:
\begin{align}
    (\ln{x})^{(n)}&=(-1)^{n-1}(n-1)!\ x^{-n},\\
    (\sin{x})^{(n)}&=\cos{\left(x+\frac{n\pi}{2}\right)}\ .
\end{align}\bigskip

\begin{defr}{Definicja 8.4: Ekstrema lokalne}
    Mówimy, że funkcja $f:X\to\R$ ma w $c\in X$ \textbf{maksimum (minimum) lokalne}, jeśli istnieje otoczenie $U\subset X$
    punktu $c$ takie, że 
    \begin{gather*}
        \forall_{x\in U}\hquad f(x)\le f(c)\\
        \left(\ \ \forall_{x\in U}\hquad f(x)\ge f(c)\ \ \right).
    \end{gather*}
\end{defr}

\begin{defr}{Definicja 8.5: Ekstrema lokalne właściwe}
    Mówimy, że funkcja $f:X\to\R$ ma w $c\in X$ \textbf{maksimum (minimum) lokalne właściwe}, jeśli istnieje otoczenie $U\subset X$
    punktu $c$ takie, że 
    \begin{gather*}
        \forall_{x\in U\setminus\set{c}}\hquad f(x)< f(c)\\
        \left(\ \ \forall_{x\in U\setminus\set{c}}\hquad f(x)> f(c)\ \ \right).
    \end{gather*}
\end{defr}

\begin{defr}{Definicja 8.6: Ekstrema globalne}
    Funkcja $f:X\to\R$ ma w $c\in X$ \textbf{maksimum (minimum) globalne}, jeśli
    \begin{gather*}
        \forall_{x\in X}\hquad f(c)\ge f(x)\\
        \left(\ \ \forall_{x\in X}\hquad f(c)\le f(x)\ \ \right).
    \end{gather*}
\end{defr}

Podobnie jak w przypadku ekstremów lokalnych, można definiować ekstrema globalne właściwe.

\begin{twier}{Twierdzenie 8.6}
    Jeżeli $f:[a,b]\to\R$ jest różniczkowalna w $c\in(a,b)$ i ma w tym punkcie ekstremum, to pochodna $f'(c)=0$.
\end{twier}

\begin{defr}{Definicja 8.7: Punkt krytyczny}
    Niech $f:x\to Y$. Mówimy, że punkt $a\in X$ jest \textbf{punktem krytycznym} funkcji $f$, jeśli funkcja $f$ nie jest różniczkowalna w punkcie $a$\footnote{Ta część definicji nie pojawiła się na wykładzie.} albo jest w tym punkcie różniczkowalna i pochodna $f'(a)=0$.
\end{defr}

Jeżeli funkcja $f$ jest różniczkowalna na całej swej dziedzinie, to wszystkie punkty, w których $f$ ma ekstrema są punktami krytycznymi. Jednakże odwrotna relacja już nie musi zachodzić. 

\begin{twier}{Twierdzenie Rolle'a 8.7}
    Niech $f:[a,b]\to\R$ będzie funkcją ciągłą, różniczkowalną na $(a,b)$, wówczas
    \begin{equation*}
        f(a)=f(b)\implies\exists_{c\in(a,b)}\ f'(c)=0.
    \end{equation*}
\end{twier}

\begin{twier}{Twierdzenie Lagrange'a o wartości średniej 8.8}
    Niech $f:[a,b]\to\R$ będzie funkcją ciągłą, różniczkowalną na $(a,b)$, wówczas
    \begin{equation*}
        \exists_{c\in(a,b)}\hquad f'(c)=\frac{f(b)-f(a)}{b-a}.
    \end{equation*}
\end{twier}

\begin{twier}{Twierdzenie Cauchy'ego o wartości średniej 8.9}
    Niech $f,g:[a,b]\to\R$ będą funkcjami ciągłymi i różniczkowalnymi na przedziale $(a,b)$, wówczas
    \begin{equation*}
        \exists_{c\in(a,b)}\hquad (f(b)-f(a))\ g'(c) = (g(b)-g(a))\ f'(c).
    \end{equation*}
    Jeśli $g'(c)\neq0$ oraz $g(a)\neq g(b)$, to powyższe zdanie można zapisać w postaci
    \begin{equation*}
        \exists_{c\in(a,b)}\hquad \frac{f(b)-f(a)}{g(b)-g(a)}=\frac{f'(c)}{g'(c)}.
    \end{equation*}
\end{twier}

Warto zwrócić tutaj uwagę na wnioski wypływające z tego twierdzenia. Jeżeli $\forall_{x\in(a,b)}\ f'(x)=0$, to funkcja $f$ jest stała na przedziale $(a,b)$. Po drugie, jeżeli $\forall_{x\in(a,b)}\ f'(x)=g'(x)$, to istnieje takie $r\in\R$, że $f(x)=g(x)+r$.

\begin{twier}{Twierdzenie 8.10}
Niech funkcja $f$ będzie różniczkowalna na $(a,b)$, wówczas
    \begin{align*}
        \forall_{x\in(a,b)}\hquad f'(x)>0&\implies\text{$f$ jest ściśle rosnąca na $(a,b)$},\\
        \forall_{x\in(a,b)}\hquad f'(x)<0&\implies\text{$f$ jest ściśle malejąca na $(a,b)$},\\
        \forall_{x\in(a,b)}\hquad f'(x)\ge0&\implies\text{$f$ jest niemalejąca na $(a,b)$},\\
        \forall_{x\in(a,b)}\hquad f'(x)\le0&\implies\text{$f$ jest nierosnąca na $(a,b)$}.
    \end{align*}
\end{twier}

\begin{twier}{Twierdzenie 8.11}
    Jeżeli funkcja $f$ jest niemalejąca i różniczkowalna na $(a,b)$, to $\forall_{x\in(a,b)}\ f'(x)\ge0$.\\
    Jeżeli funkcja $f$ jest nierosnąca i różniczkowalna na $(a,b)$, to $\forall_{x\in(a,b)}\ f'(x)\le0$.
\end{twier}

\begin{twier}{Twierdzenie 8.12}
    Niech $f:(a,b)\to\R$ oraz $c\in(a,b)$. Jeśli 
    \begin{enumerate}
        \item $f'(c)>0\hquad (f'(c)<0$),
        \item $f'$ istnieje na pewnym otoczeniu $c$ i jest ciągła w $c$,
    \end{enumerate}
    to na tym otoczeniu punktu $c$ funkcja $f$ jest ściśle rosnąca (malejąca).
\end{twier}

\begin{twier}{Twierdzenie 8.13}
    Niech $f:\R\supset X\to\R$ oraz $c\in X$. Jeśli $f'(c)\neq0$, $f'$ istnieje na otoczeniu punktu $c$ i jest na nim ciągła, to istnieje otoczenie punktu $c$, takie że $g:(c-\delta,c+\delta)\to f[(c-\delta,c+\delta)]$ jest bijekcją.
\end{twier}

\begin{defr}{Definicja 8.8: Homomorfizm}
    Ciągłą bijekcję, której odwrotność również jest ciągła, nazywamy \textbf{homomorfizmem}.
\end{defr}

\begin{defr}{Definicja 8.9: Zbiory homomorficzne}
    Zbiory pomiędzy, którymi istnieje homomorfizm nazywamy \textbf{homomorficznymi}.
\end{defr}

\begin{twier}{Reguła de l'Hospitala 8.14}
    \hfill \texttt{WERSJA I}\\ \smallskip
Niech $f,g$ będą funkcjami rzeczywistymi, różniczkowalnymi na $(a,b)$ i $\forall_{x\in(a,b)}\ g'(x)\neq0$, gdzie $-\infty\le a<b\le+\infty$ oraz
\begin{gather*}
   \left( \lim_{x\to a^+}f(x)=0\land \lim_{x\to a^+}g(x)=0\right)\ \lor\ \left( \lim_{x\to a^+}f(x)=\pm\infty\land \lim_{x\to a^+}g(x)=\pm\infty\right),\ \text{wówczas}\\
   \left(\exists_{g\in\closure{\R}}\ \lim_{x\to a^+}{\frac{f'(x)}{g'(x)}}=g\right)\implies\lim_{x\to a^+}{\frac{f(x)}{g(x)}}=g.
\end{gather*}
Analogiczne twierdzenie jest też prawdziwe dla granic lewostronnych i obustronnych.
\bigskip

    \hfill \texttt{WERSJA II}\\ \smallskip
Niech $f,g$ będą funkcjami rzeczywistymi, różniczkowalnymi na $(c,+\infty)$ i $\forall_{x\in(c,+\infty)}\ g'(x)\neq0$, gdzie $-\infty\le c<+\infty$ oraz
\begin{gather*}
    \left( \lim_{x\to +\infty}f(x)=0\land \lim_{x\to +\infty}g(x)=0\right)\ \lor\ \left( \lim_{x\to +\infty}f(x)=\pm\infty\land \lim_{x\to +\infty}g(x)=\pm\infty\right),\ \text{wówczas}\\
    \left(\exists_{g\in\closure{\R}}\ \lim_{x\to +\infty}{\frac{f'(x)}{g'(x)}}=g\right)\implies\lim_{x\to +\infty}{\frac{f(x)}{g(x)}}=g.
\end{gather*}
Analogiczne twierdzenie jest też prawdziwe dla granic gdy $x\to-\infty$.
\end{twier}

Zbiór $\closure{\R}$ definiujemy jako $\R\cup\set{-\infty,+\infty}$. Chcąc zastosować regułę de l'Hospitala do funkcji postaci $f(x)^{g(x)}$, można je przedstawić jako $e^{g(x)\ln{f(x)}}$.

\begin{twier}{Wzór Leibniza 8.15}
    Niech $f,g$ będą funkcjami różniczkowalnymi, posiadającymi pochodne do rzędu $n$ włącznie. Wówczas pochodna $n$-tego rzędu iloczynu $f\cdot g$ wyraża się wzorem:
    \begin{equation*}
        (f\cdot g)^{(n)}=\sum_{k=0}^{n}{f^{(n-k)}g^{(k)}}.
    \end{equation*}
\end{twier}

\begin{defr}{Definicja 8.10: Funkcja wypukła}
    Niech $f:A\to\R$, gdzie $A$ jest przedziałem liczb rzeczywistych. Funkcję $f$ nazywamy \textbf{wypukłą} na przedziale $A$, jeśli 
    \begin{equation*}
        \forall_{x_1,x_2\in A}\ \forall_{\theta\in[0,1]}\hquad f(\theta x_1+(1-\theta)x_2)\le\theta f(x_1)+(1-\theta)f(x_2).
    \end{equation*}
\end{defr}

\begin{defr}{Definicja 8.10: Funkcja wklęsła}
    Niech $f:A\to\R$, gdzie $A$ jest przedziałem liczb rzeczywistych. Funkcję $f$ nazywamy \textbf{wklęsłą} na przedziale $A$, jeśli 
    \begin{equation*}
        \forall_{x_1,x_2\in A}\ \forall_{\theta\in[0,1]}\hquad f(\theta x_1+(1-\theta)x_2)\ge\theta f(x_1)+(1-\theta)f(x_2).
    \end{equation*}
\end{defr}

Zazwyczaj w definiowaniu funkcji wypukłej i wklęsłej korzysta się z pojęcia zbioru wypukłego, jednak na wykładzie otrzymaliśmy węższą definicję. Przedziały liczb rzeczywistych są zbiorami wypukłymi.

\begin{twier}{Twierdzenie 8.16}
    Funkcja $f:A\to\R$ jest wypukła (wklęsła) na zbiorze liczb rzeczywistych $A$ \textit{wtedy i tylko wtedy, gdy}
    \begin{gather*}
        \forall_{n\in\N,\, n\ge2}\ \forall_{x_1,\dots,x_n\in A}\ \forall_{\theta_1,\dots,\theta_2\in[0,1]\hquad }f\left(\sum_{i=1}^{n}{\theta_{i}x_{i}}\right)\le\sum_{i=1}^{n}{\theta_{i}f(x_{i})}\\
        \left(\ \forall_{n\in\N,\, n\ge2}\ \forall_{x_1,\dots,x_n\in A}\ \forall_{\theta_1,\dots,\theta_2\in[0,1]\hquad } f\left(\sum_{i=1}^{n}{\theta_{i}x_{i}}\right)\ge\sum_{i=1}^{n}{\theta_{i}f(x_{i})}\ \right).
    \end{gather*}
\end{twier}

\begin{twier}{Twierdzenie 8.17}
    Dana jest funkcja $f:A\to\R$, gdzie $A$ jest przedziałem liczb rzeczywistych. Niech $\alpha_{ij}=\frac{f(x_i)-f(x_j)}{x_i-x_j}$. Mówimy, że $f$ jest wypukła \textit{wtedy i tylko wtedy, gdy}
    \begin{itemize}
        \item[($\ifff$)] $\forall_{x_1,x_2,x_3\in A,\,x_1<x_2<x_3}\hquad \alpha_{21}\le\alpha_{32}$,
        \item[($\ifff$)] $\forall_{x_1,x_2,x_3\in A,\,x_1<x_2<x_3}\hquad \alpha_{31}\le\alpha_{32}$,
        \item[($\ifff$)] $\forall_{x_1,x_2,x_3\in A,\,x_1<x_2<x_3}\hquad \alpha_{21}\le\alpha_{31}$.
    \end{itemize}
\end{twier}

\begin{wn}{Wnioski}
\begin{enumerate}
    \item Jeżeli funkcja $f$ jest wypukła na $(a,b)$, to $\forall_{x\in(a,b)}$ istnieją pochodne jednostronne, ponadto $f'_{+}(x)\ge f'_{-}(x)$.
    \item Funkcja wypukła na przedziale otwartym jest ciągła.
    \item Funkcja $f$ jest różniczkowalna i wypukła na przedziale $(a,b)$ \textit{wtedy i tylko wtedy, gdy} jej pochodna $f'$ jest niemalejąca.
    \item Jeżeli funkcja $f$ jest dwukrotnie różniczkowalna na $(a,b)$, to $f$ jest wypukła na $(a,b)$ \textit{wtedy i tylko wtedy, gdy} $\forall_{x\in(a,b)\ f''(x)\ge0}$.
\end{enumerate}
\end{wn}

\begin{twier}{Wzór Taylora z resztą w postaci Peano 8.18}
    Niech $f:(a,b)\to\R$ i $x_0\in(a,b)$. Jeśli $f$ ma $(n-1)$ pochodnych na $(a,b)$ i $n$-tą pochodną w punkcie $x_0$, to
    \begin{equation*}
        f(x)=\sum_{k=0}^{n}{\frac{f^{(k)}(x_0)}{k!}(x-x_0)^k}+R_n(x,x_0),\quad \lim_{x\to x_0}{\frac{R_n(x,x_0)}{(x-x_0)^i}}=0.
    \end{equation*}
\end{twier}

Zamiast podawać warunek po przecinku, pisze się czasem $R_n(x,x_0)=o(x^i)$ dla $x\to0$. Symbol ten to \textit{o małe}; napis $f(x)=o(g(x))$ dla $x\to a$ oznacza, że $\lim_{x\to a}{f(x)/g(x)}=0$. Wzór Taylora z resztą w postaci Peano dla $x_0=0$ nazywamy \textit{wzorem Maclaurina}.

\begin{twier}{Wzór Taylora z resztą w postaci Cauchy'ego i Lagrange'a 8.19}
    Niech $f:(a,b)\to\R$ ma w przedziale $(a,b)$ pochodne rzędu $(n+1)$ włącznie, wówczas\medskip

    \hfill \texttt{z resztą w postaci Lagrange'a}\\ 
    \begin{equation*}
        \forall_{x,x_0\in(a,b)}\ \exists_{\theta\in(0,1)}\hquad f(x)=\sum_{k=0}^{n}{\frac{f^{(k)}(x_0)}{k!}(x-x_0)^k}+\frac{(x-x_0)^{n+1}}{(n+1)!}f^{(n+1)}(x_0+\theta(x-x_0)),
    \end{equation*}

    \hfill \texttt{z resztą w postaci Cauchy'ego}\\
    \begin{equation*}
        \forall_{x,x_0\in(a,b)}\ \exists_{\theta\in(0,1)}\hquad f(x)=\sum_{k=0}^{n}{\frac{f^{(k)}(x_0)}{k!}(x-x_0)^k}+\frac{(x-x_0)^{n+1}(1-\theta)^{n}}{n!}f^{(n+1)}(x_0+\theta(x-x_0)).
    \end{equation*}
\end{twier}

\begin{table}[ht]
\begin{center}
    \begin{tabular}{|c|c|}
    \hline
    $f(x)$                              & $f'(x)$                                         \\ \hline
    $x^\alpha$                          & $\alpha x^{\alpha-1}$                           \\ \hline
    $a^x$                               & $a^x\,\ln{a}$                                   \\ \hline
    $\log_{a}{x}$                       & $1/(x\ln{a})$                                   \\ \hline
    $\sin{x}$                           & $\cos{x}$                                       \\ \hline
    $\cos{x}$                           & $-\sin{x}$                                      \\ \hline
    $\tan{x}$                           & $1/\cos^2{x}$                                   \\ \hline
    $\arcsin{x}$                        & $1/\sqrt{1-x^2}$                                \\ \hline
    $\arccos{x}$                        & $-1/\sqrt{1-x^2}$                               \\ \hline
    $\arctan{x}$                        & $1/(1+x^2)$                                     \\ \hline
    $\sinh{x}$                          & $\cosh{x}$                                      \\ \hline
    $\tanh{x}$                          & $1-\tanh^2{x}$                                  \\ \hline
    $\arcsinh{x}$                       & $1/\sqrt{x^2+1}$                                \\ \hline
    \multicolumn{1}{|l|}{$\arccosh{x}$} & \multicolumn{1}{l|}{$1/(\sqrt{1-x}\sqrt{1+x})$} \\ \hline
    \multicolumn{1}{|l|}{$x^x$}         & \multicolumn{1}{l|}{$x^x(\ln{x}+1)$}            \\ \hline
    \multicolumn{1}{|l|}{$|h(x)|$}      & \multicolumn{1}{l|}{$\sgn{(h(x))}\ h'(x)$}        \\ \hline
    \end{tabular}
\end{center}
\end{table}


















                                                \end{document}
